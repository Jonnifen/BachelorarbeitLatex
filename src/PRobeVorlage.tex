%! Author = jbf
%! Date = 25.07.25

% Preamble
\documentclass[
pagesize,				% flexible Blattgröße
a4paper,				% DIN A4
oneside,				% einseitig gedruckt
headsepline,		    % Strich unter der Kopfzeile
11pt,					% 12pt Schriftgröße
halfparskip,		    % Europäischer Satz: Abstand zwischen Absätzen
%draft,					%Vorabversion
final,					% Endgültige Version
listof=totoc,		    % nicht nummerierter eintrag in den Inhalt (verzeichnisse)
]{scrartcl}			    % KOMA_Scriptklasse Artikel

% Packages und Grundeinstellungen des Dokumentes wurde ausgelagert

\usepackage[english,ngerman]{babel}		    %deutsche Trennmuster
\usepackage[utf8]{inputenc}					%direkte Umlauteingabe
\usepackage[T1]{fontenc}
\usepackage{mathptmx}	                    %Schriftpaket\usepackage{newtxtext,newtxmath}      % Times-artige Schrift inkl. Mathe
\usepackage[scaled]{helvet}
\renewcommand{\familydefault}{\sfdefault}
\usepackage{array,ragged2e}					%wichtig für Abstandsformatierung
\usepackage{tocbasic} % KOMA für Verzeichnisse

\usepackage[automark,plainheadsepline=on]{scrlayer-scrpage}				%Anpassung von Kopf- und Fusszeilen
\usepackage{xspace}														%Korrektur Leerraum nach Befehlsdefinitionen
\usepackage{setspace}													%Anderthalbzeiliger Abstand
\usepackage{graphicx}											        %zum Einbinden von Grafiken
\usepackage[absolute,overlay]{textpos}									%Boxen absolut Positionieren
\usepackage[final]{pdfpages}											%Einfügen von externen PDF Dokumenten
\usepackage{natbib}														%Neuimplementierung des \cite-Kommandos

\usepackage[
  top=3cm,      % Abstand oben
  left=3.5cm,   % Abstand links
  right=2cm,    % Abstand rechts
  bottom=2cm    % Abstand unten
]{geometry}

\usepackage{upgreek}            %Ermöglicht Aufrechte Varianten von Griechischen Buchstaben


 \usepackage[colorlinks,		% Einstellen und Laden des Hyperref-Pakets
	pdftex,
	bookmarks,
	bookmarksopen=false,
	bookmarksnumbered,
	citecolor=black,
	linkcolor=black,
	urlcolor=black,
	filecolor=black,
	linktocpage,
  pdfstartview=Fit,                  % startet mit Ganzseitenanzeige
	pdfsubject={Datenverarbeitung und Visualisierung von Umrichter-Testbench-Daten},
	pdftitle={Bachelorthesis im Fachbereich Elektrotechnik und Inforationstechnik an der FH Westküste},
	pdfauthor={Jon Feddersen}]{hyperref}

\usepackage{listings}


\setcounter{secnumdepth}{4}     %Nummerierung bis paragraph aktivieren
\setcounter{tocdepth}{4}        %Inhaltsverzeichnis zeigt bis \paragraph

\graphicspath{{Grafiken/}}      %Vereinfach das Einbinden von Grafiken da davon ausgegangen wird das alle in /Grafiken sind

\setlength{\parindent}{0pt} % Einrücken nach Absatzänderung


% Document
\begin{document}

\newpage
\section*{Sperrvermerk}
\thispagestyle{empty}
Diese Arbeit enthält vertrauliche Daten und Informationen des Unternehmens, in dem die Bachelor-/Masterarbeit angefertigt wurde.
Sie darf Dritten deshalb nicht zugänglich gemacht werden. \\
Die für die Prüfung notwendigen Exemplare verbleiben beim Prüfungsamt und beim betreuenden Hochschullehrer.
\newpage




%Definition von Großen römischen Zahlen für die Kapitel
\renewcommand{\thesection}{\roman{section}}
\pagenumbering{Roman}
\ohead[\pagemark]{}


\singlespacing
\renewcommand{\contentsname}{Inhaltsverzeichnis}
\phantomsection
\addtocounter{section}{1}
\setcounter{page}{2}
\clearscrheadings
\clearscrplain
\clearscrheadfoot
\ohead[\pagemark]{\pagemark}
\ihead[\headmark]{\headmark}
\tableofcontents
\pagebreak

\listoffigures
\pagebreak

\listoftables
\pagebreak


%Definition für Kopfzeile und arabische Sections
\renewcommand{\sectionmark}[1]{\markright{#1}}
\renewcommand{\subsectionmark}[1]{}
\renewcommand{\subsubsectionmark}[1]{}
\onehalfspacing
\renewcommand{\thesection}{\arabic{section}}
\setcounter{section}{0}
\pagenumbering{arabic}
\setcounter{page}{1}

%------------------------------------------------
%1. Einleitung in die Arbeit
%------------------------------------------------


\newpage
\section{Einleitung}
\label{Einleitung}
In der modernen Antriebstechnik spielen Umrichter eine zentrale Rolle bei der Steuerung und Regelung elektrischer Maschinen.
Um ihre Leistungsfähigkeit und Zuverlässigkeit zu gewährleisten, werden sie in Testbenches unter verschiedenen Betriebsbedingungen geprüft.
Dabei entstehen große Mengen an Messdaten, die oft in XML-Formaten vorliegen.
Die manuelle Auswertung dieser Rohdaten ist zeitaufwendig, fehleranfällig und erschwert die schnelle Identifikation relevanter Muster oder Anomalien.
Besonders bei komplexen Testläufen kann der fehlende direkte Zugriff auf übersichtlich aufbereitete Ergebnisse den Entwicklungsprozess verlangsamen.
Ziel dieser Arbeit ist es, eine Web-Applikation zu entwickeln, die XML-Daten aus Umrichter-Testbenches automatisiert einliest, in einer Datenbank speichert und interaktiv visualisiert.
Dadurch soll die Auswertung vereinfacht, die Datenanalyse beschleunigt und die Entscheidungsfindung im Entwicklungsprozess unterstützt werden.
Die Lösung wird mit modernen Webtechnologien umgesetzt und legt den Fokus auf eine effiziente Datenverarbeitung, flexible Filtermöglichkeiten sowie eine intuitive Benutzeroberfläche.


%------------------------------------------------
%2. Hauptteil in die Arbeit
%------------------------------------------------

\newpage
\section{Grundlagen}
\label{Grundlagen}

\subsection{Überblick über den Umrichter-Prüfstand}

\subsection{Verarbeitung von XML-Daten}

\subsection{Datenbankentwurf und Normalisierung}

\subsection{Grundlagen der Datenvisualisierung}

\subsection{Anforderungen an modulare Softwareentwicklung}

%------------------------------------------------

\newpage
\section{Analyse und Konzeption}
\label{Analyse und Konzeption}

\subsection{Defintion der Anforderungen}

\subsection{Analyse der bestehenden Strukturen und Prozesse}

\subsection{Datenbankdesign und Strukturkonzeption}

\subsection{Grundkonzept des Benutzeroberflächen-Design}

\subsection{Entwurf der Applikationsarchitektu}

%------------------------------------------------

\newpage
\section{Implementierung}
\label{Implementierung}

\subsection{Einlesen und Verarbeiten von XML-Daten}

\subsection{Implementierung der Datenbank}

\subsection{Entwicklung der Benutzeroberfläche}

\subsection{Technische Details zur Visualisierung}

%------------------------------------------------

\newpage
\section{Integration und Test}
\label{Integration und Test}

\subsection{Einbindung in die bestehende Systemlandschaft}

\subsection{Testmethoden und Durchführung}

\subsection{Ergebnisse der Testmethoden}

%------------------------------------------------
%3. Schluss in die Arbeit
%------------------------------------------------

\newpage
\section{Fazit und Ausblick}
\label{Fazit und Ausblick}

\subsection{Zusammenfassung der Ergebnisse}

\subsection{Kritische Bewertung}

\subsection{Möglichkeiten für zukünftige Erweiterungen}
\newpage
%------------------------------------------------
%Literaturverzeichnis, Anhang und Erklärung
%------------------------------------------------
%Einstellungsänderungen der Seitenzahlen
\renewcommand{\thesection}{\roman{section}}
\pagenumbering{roman}
\clearscrheadings
\clearscrplain
\clearscrheadfoot
\ohead[\pagemark]{\pagemark}
\ihead[\headmark]{\headmark}
%------------------------------------------------

\section*{Erklärung}
\addcontentsline{toc}{section}{Erklärung}
\label{Erklärung}
Hiermit erkläre ich, dass ich die von mir eingereichte Bachelor- / Masterarbeit
"......(Titel der Arbeit)........" selbständig und nur unter Verwendung der
angegebenen Quellen und Hilfsmittel angefertigt habe.
\\
Ort und Datum
\\
\\
persönliche Unterschrift
\\
\\
\\
\\
(Name des Verfassers)


\end{document}


