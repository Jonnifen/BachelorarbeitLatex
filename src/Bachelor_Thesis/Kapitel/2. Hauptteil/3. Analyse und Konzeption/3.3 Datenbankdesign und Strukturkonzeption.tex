\subsection{Datenbankdesign und Strukturkonzeption}
\label{subsec:datenbankdesign-und-strukturkonzeption}

In diesem Unterkapitel soll der Verlauf der Erstellung der Datenbank beschrieben und nachvollzogen werden. Es wurde in
den Anforderungen festgelegt, dass MariaDB genutzt werden soll. MariaDB ist ein kostenloses, relationales
Open-Source-Datenbankmanagementsystem, das aus einer Abspaltung von MySQL entwickelt wurde. MySQLs früherer
Hauptentwickler Michael Widenius hat das Projekt ins Leben gerufen. \cite{mariadb.org}

\subsubsection{Betrachtung der Anforderungen}

Für das genaue Interpretieren und Umsetzen der Anforderungen wurde festgelegt, dass die Ersteller/-in der Datenbank erst
ein Konzept nach den festgelegten Anforderungen erarbeiten und umsetzen.
Diese Umsetzung wird dann im Laufe der weiteren Erstellung der Applikation, falls nötig, angepasst.

\subsubsection{Konzeptionelles Datenmodell}

Für das Erstellen der Konzeption des Datenbankmodelles wurde die XML-Struktur analysiert, um die Kenntnisse und Muster
aus Unterkapitel \ref{subsec:analyse-der-generierten-xml-berichte-und-bestehenden-strukturen} abzuleiten.
Aus diesen Erkenntnissen wurde dann ein ER-Modell erstellt, siehe Abbildung \ref{fig: ER-Modell Überlegung der Datenbankstruktur}.
Im Folgenden wird das ER-Modell kurz erläutert.

Durch das Durchführen eines Tests wird ein Bericht erstellt. Dieser Bericht enthält einige Attribute, welche
Testinformationen, darunter die Seriennummern, das Datum und die Uhrzeit des Tests, die Materialnummer (DUT-ID), die
Konfigurations-Version und die Carrier-Bezeichnung enthalten. Zudem besitzt der Bericht ein Unterelement „Testbenchinformationen“, das die Hard- und Softwareversionen des Teststandes enthält. In der Unterentität Testmodelle
befinden sich Versions-, Zeit- und Ergebnisinformationen sowie drei Datensätze, welche die Testparameter und die Messdaten
enthalten. Diese Datensätze können den Phasen des Stromnetzes des Teststandes und somit den DUTs zugeordnet werden.
Zusätzlich zu dem Aufbau wurden hier schon Schlüsselattribute hinzugefügt, die später als Primärschlüssel der
Datenbanktabellen dienen sollen, orange markiert.

\begin{figure}[H]
    \centering
    \includegraphics[width=0.95\textwidth]{Grafiken/Bild von ER-Modell}
    \caption{ER-Modell Überlegung der Datenbankstruktur}
    \label{fig: ER-Modell Überlegung der Datenbankstruktur}
    {Quelle: Eigene Darstellung mit Microsoft Visio}
\end{figure}

Aus diesem ER-Model wurde die Datenbanktabelle und ihre Verbindungen über Fremdschlüssel abgeleitet. In Abbildung \ref{fig: Darstellung der Datenbanktabellen und ihrer Verbindungen}
werden die Datenbanktabellen, ihre genauen Namensbezeichnungen der Tabelle, der Datentyp in der Datenbank und ihr Inhalt sowie ihre Verbindungen abgebildet.

\begin{figure}[H]
    \centering
    \includegraphics[width=0.95\textwidth]{Grafiken/Tabellendiagramm Datenbank}
    \caption{Darstellung der Datenbanktabellen und ihrer Verbindungen}
    \label{fig: Darstellung der Datenbanktabellen und ihrer Verbindungen}
    {Quelle: Eigene Darstellung mit Microsoft Visio}
\end{figure}


Zur persistenten Datenspeicherung wird das ORM-Framework \textit{Flask-SQLAlchemy} verwendet, das auf \textit{SQLAlchemy} basiert.
Es ermöglicht die objektorientierte Abbildung relationaler Datenbanken und vereinfacht den Zugriff auf Daten über Python-Klassen.
Dadurch wird eine saubere Trennung von Anwendungslogik und Datenzugriffsschicht erreicht.


