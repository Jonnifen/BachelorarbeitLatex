\subsection{Entwurf der Applikationsarchitektur}
\label{subsec:entwurf-der-applikationsarchitektur}


Der Entwurf der Applikationsarchitektur bildet die konzeptionelle Grundlage für die technische Umsetzung der entwickelten Web-Applikation.
Das Ziel ist es, eine Struktur zu schaffen, die modular, wartbar und erweiterbar ist, und die gleichzeitig den funktionalen Anforderungen gerecht wird und sich in die bestehende Systemlandschaft integrieren lässt.
Die Architektur der Anwendung ist schichten- und komponentenorientiert aufgebaut und nutzt das Webframework Flask, welches eine klare Trennung zwischen der Präsentations-, Logik- und Datenhaltungsschicht ermöglicht.


\subsubsection{Architekturübersicht}

Die Applikation soll als serverbasierte Webanwendung im Client-Server-Modell funktionieren.
Das bedeutet: Während der Server die Datenverarbeitung, -speicherung und -bereitstellung übernimmt, ermöglicht der Client über den Webbrowser die Darstellung und Interaktion.
Die Anwendung ist in mehrere Schichten gegliedert, zu denen man die einzelnen Programmteile zuordnen kann.
Die Anwendung wird in folgende Schichten gegliedert:

\begin{itemize}
\item Präsentationsschicht (Frontend): Bereitstellung der Benutzeroberfläche über HTML-Templates, CSS und JavaScript.

\item
Applikationslogik (Backend): Implementierung der Geschäftslogik, Steuerung des Datenflusses und Verarbeitung der XML-Dateien.

\item
Datenhaltungsschicht: Persistente Speicherung der extrahierten Mess- und Gerätedaten in einer relationalen Datenbank mittels ORM.

\end{itemize}

Ein schematisches Architekturdiagramm dieser Struktur ist in Abbildung XX dargestellt.



