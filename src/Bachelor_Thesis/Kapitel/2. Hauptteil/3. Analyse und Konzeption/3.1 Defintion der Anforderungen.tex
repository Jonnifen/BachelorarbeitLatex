\subsection{Defintion der Anforderungen}
\label{subsec:defintion-der-anforderungen}

In diesem Unterkapitel sollen die gestellten Anforderungen der verschieden Pathein erläutert werden, diese werden in technische und Nutzer-anforderungen aufgeteilt.
Die technischen Anforderungen werden den Rahmen für die verwendete Hard- und Software bestimmen.
Die Nutzeranforderungen beschreiben die Anforderungen der Verwenderinnen der Apploikation, so wie der Teststand bedienenden Personen.

\subsubsection{Technische Anforderungen}


\begin{enumerate}

    \item die Web-Applikation soll auf einem Apache2 Server der Firma laufen.
    \item Es wurde festgelegt, das für die Frontendentwicklung mit HTML und CSS durchgeführt werden soll.
    \item Das Backend soll mit Python programmiert werde.
    \item Für die Visualisierung und die Kommunikation von Frond und Backend soll JavaScirpt genutzt werden.
    \item Als Webframework soll Flask verwendet werden, da es sehr minimalistisch und Python basiert ist.
    \item Bezüglich der Geschwindigkeit und Leistungsstärke würden keine genauen Anforderungen definiert,
    die Applikation möglichst effizient Arbeiten, es muss jedoch keine aktive Optimierung erfolgen so
    lange der Arbeitsfluss nicht stark behindert wird.
\end{enumerate}

\subsubsection{Nutzeranforderungen}

\begin{enumerate}

    \item
    \item
    \item

\end{enumerate}
