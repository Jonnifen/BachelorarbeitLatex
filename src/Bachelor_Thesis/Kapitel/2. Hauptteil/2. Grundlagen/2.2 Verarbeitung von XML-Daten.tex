\subsection{Verarbeitung von XML-Daten}
\label{subsec:verarbeitung-von-xml-daten}

Dieses Kapitel behandelt einige Grundlegende und für dies Arbeit relevante Aspekte von dem Dokumententype XML,
da die Messwerte bzw. die Berichte die der Testand generiert im diesem Format vorliegen.

\ac{XML}
\color{red}     %Unfertig
XML, die Abkürzung für , ist eine der beliebtesten Auszeichnungssprachen, die entwickelt wurde,
um Informationen in einem maschinenlesbaren und strukturierten Format zu speichern und zu transportieren.
Es ist eine textbasierte Auszeichnungssprache, die heute in vielen Anwendungen verwendet wird, wie z.B. Webdiensten,
Datenbanken, Konfigurationsdateien und vielen anderen.
XML ermöglicht die hierarchische Organisation von Informationen in einer strukturierten Weise, die sowohl für Menschen
als auch für Maschinen verständlich ist.

Die Motivation hinter XML war, eine universelle und erweiterbare Sprache zu schaffen, die von verschiedenen Systemen
unabhängig von der zugrunde liegenden Technologie genutzt werden kann.
Die Fähigkeit, Daten in einem offenen Standard zu speichern und auszutauschen, war entscheidend,
um die Interoperabilität zwischen verschiedenen Anwendungen und Plattformen zu fördern.

\subsubsection{XML-Strukturaufbau}

Ein XML-Dokument besteht aus einer Reihe von Elementen, die durch Tags markiert sind.
Die grundlegenden Bestandteile eines XML-Dokuments sind:

Prolog: Ein optionaler Abschnitt, der die XML-Version und die verwendete Zeichencodierung definiert.
Ein typischer Prolog sieht so aus:

\begin{figure}[H]
\centering
\begin{minipage}{0.95\textwidth}
\begin{lstlisting}[language=XML]
<?xml version="1.0" encoding="UTF-8"?>
\end{lstlisting}
\end{minipage}
\caption{XML Prolog Beispielcode}
\label{fig:XML Prolog Beispielcode}
    {Quelle: eigene Darstellung}
\end{figure}

Elemente: Die eigentlichen Daten werden in Elementen gespeichert, die mit einem Start-Tag beginnen und mit einem End-Tag abschließen.
Ein Element kann weitere Elemente enthalten, was zu einer hierarchischen Struktur führt:

\begin{figure}[H]
\centering
\begin{minipage}{0.95\textwidth}
\begin{lstlisting}[language=XML]
<buch>
  <titel>XML-Grundlagen</titel>
  <autor>Max Mustermann</autor>
</buch>
\end{lstlisting}
\end{minipage}
\caption{XML Elemente Beispielcode}
\label{fig:XML Elemente Beispielcode}
    {Quelle: eigene Darstellung}
\end{figure}

Attribute: Jedes Element kann Attribute haben, die zusätzliche Informationen enthalten.
Attribute werden im Start-Tag eines Elements definiert:

\begin{figure}[H]
\centering
\begin{minipage}{0.95\textwidth}
\begin{lstlisting}[language=XML]
<buch genre="Lehrbuch">
  <titel>XML-Grundlagen</titel>
  <autor>Max Mustermann</autor>
</buch>
\end{lstlisting}
\end{minipage}
\caption{XML Attribute Beispielcode}
\label{fig:XML Attribute Beispielcode}
    {Quelle: eigene Darstellung}
\end{figure}

Kommentare: Kommentare werden mit den Tags <!-- und --> eingefügt und dienen der Dokumentation oder dem Hinweis auf bestimmte Teile des Codes,
Sie werden beim Parsen des Dokuments ignoriert. In Abbildung \ref{fig:XML Kommentare Beispielcode} ist ein Beispiel wie
ein Kommentar genutzt wird.

\begin{figure}[H]
\centering
\begin{minipage}{0.95\textwidth}
\begin{lstlisting}[language=XML]
<!-- Dies ist ein Kommentar -->
\end{lstlisting}
\end{minipage}
\caption{XML Kommentare Beispielcode}
\label{fig:XML Kommentare Beispielcode}
    {Quelle: eigene Darstellung}
\end{figure}

Textinhalt: Zwischen den Start- und End-Tags eines Elements kann Text enthalten sein:

\begin{figure}[H]
\centering
\begin{minipage}{0.95\textwidth}
\begin{lstlisting}[language=XML]
<titel>XML-Grundlagen</titel>
\end{lstlisting}
\end{minipage}
\caption{XML Text Beispielcode}
\label{fig:XML Text Beispielcode}
    {Quelle: eigene Darstellung}
\end{figure}

\subsubsection{Verarbeiten von XML-Dateien mit Python}
XML (Extensible Markup Language) ist ein vielseitiges und weit verbreitetes Format zum Speichern und Austauschen von strukturierten Daten.
Es wird in zahlreichen Anwendungsbereichen verwendet, darunter Web-Dienste, Datenbanken, Konfigurationsdateien und mehr.
Die Fähigkeit, XML-Daten zu verarbeiten, ist daher eine Schlüsselkompetenz in der modernen Softwareentwicklung.
In diesem Kapitel werden die grundlegenden Methoden und Techniken zur Verarbeitung von XML-Dateien in verschiedenen Programmiersprachen und mit unterschiedlichen Werkzeugen vorgestellt.
Es wird erläutert, wie XML-Dateien geparst, bearbeitet und validiert werden können, um sie für verschiedene Anwendungen nutzbar zu machen.

Parsing von XML-Daten

Das Parsing von XML-Daten ist der erste Schritt bei der Verarbeitung.
Dabei wird die XML-Datei in ein Programm geladen und in eine für das Programm verständliche Struktur überführt.
Es gibt mehrere Methoden, um XML-Daten zu parsen, je nach Anforderungen der Anwendung.

DOM (Document Object Model)

Das DOM ist ein objektorientiertes Modell zur Darstellung eines XML-Dokuments.
Es stellt das gesamte Dokument als Baumstruktur dar, wobei jedes Element, Attribut und der Text als Knoten im Baum betrachtet werden.
Der Vorteil von DOM ist, dass es eine vollständige in-memory Repräsentation des XML-Dokuments bietet, was das Durchsuchen und Bearbeiten des Dokuments einfach macht.
DOM hat den Vorteil, dass es eine einfache Schnittstelle für die Manipulation und das Durchsuchen von XML-Daten bietet.
cAllerdings ist es speicherintensiv, da es das gesamte Dokument in den Arbeitsspeicher lädt, was bei sehr großen XML-Dateien problematisch sein kann.

SAX (Simple API for XML)

Im Gegensatz zu DOM ist SAX ein ereignisbasierter Parser.
SAX liest die XML-Datei sequenziell und löst Ereignisse aus, wenn es auf bestimmte Tags oder Text stößt.
Es lädt das XML-Dokument nicht vollständig in den Speicher, sondern verarbeitet es während des Lesens.
Dies macht SAX besonders nützlich für die Verarbeitung großer XML-Dateien.
SAX ist effizienter in Bezug auf den Speicherverbrauch, eignet sich jedoch weniger für die Manipulation von Daten, da es keine vollständige Dokumentstruktur aufbaut.

ElementTree

ElementTree ist eine einfache und leichtgewichtige Python-Bibliothek zur Verarbeitung von XML. Sie stellt eine Baumstruktur zur Verfügung,
die das XML-Dokument effizient repräsentiert und einfache Methoden zum Navigieren, Bearbeiten und Erstellen von XML-Dokumenten bietet.
ElementTree ist sowohl speichereffizient als auch einfach zu verwenden und eignet sich daher gut für die meisten Anwendungen, bei denen XML-Daten nicht zu groß sind.

Validierung von XML-Daten

Die Validierung von XML-Daten ist ein wichtiger Schritt, um sicherzustellen, dass die Daten einer vordefinierten Struktur entsprechen.
Dies kann mit Hilfe von XML-Schemas (XSD) erfolgen.

Validierung mit XML-Schema
XML-Schema ist eine Methode, um die Struktur von XML-Dokumenten zu definieren.
Es ermöglicht das Validieren von XML-Daten, um sicherzustellen, dass sie die richtigen Elemente, Attribute und Datentypen enthalten.
In Python kann das lxml-Modul zur Validierung von XML-Daten gegen ein Schema verwendet werden.


\pagebreak