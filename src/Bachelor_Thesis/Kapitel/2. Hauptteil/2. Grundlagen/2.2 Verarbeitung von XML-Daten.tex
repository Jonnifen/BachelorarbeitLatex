\subsection{Verarbeitung von XML-Daten}
\label{subsec:verarbeitung-von-xml-daten}

Dieses Kapitel behandelt einige Grundlegende und für dies Arbeit relevante Aspekte von dem Dokumententype \ac{XML},
da die Messwerte bzw. die Berichte die der Testand generiert im diesem Format vorliegen.


\ac{XML} ist eine Auszeichnungssprachen, also eine
formale Sprachen, die verwendet werden, um die Struktur und Darstellung von Daten oder Texten zu beschreiben \cite*{Neu},
die entwickelt wurde, um Informationen in einem maschinenlesbaren und strukturierten Format zu speichern und zu übermitteln.
Sie wird hauptsächlich in Bereichen wie Webdiensten, Datenbanken, Konfigurationsdateien eingesetzt.
\ac{XML} ermöglicht die hierarchische Organisation von Informationen in einem strukturierten Aufbau, die sowohl für Menschen
als auch für Maschinen verständlich ist.

Das Grundkonzept hinter \ac{XML} war, eine universelle einsetzbare und erweiterbare Sprache zu erschaffen, die von verschiedenen Systemen
unabhängig von deren grundlegenden Technologieansatz genutzt werden kann.
Hierbei ware das angestrebte Ziel Daten in einem einheitlichen Standard zwischen verschiedenen Anwendungen und Plattformen zu speichern und auszutauschen
zu kömmen.

\subsubsection{XML-Strukturaufbau}

Ein \ac{XML}-Dokument besteht aus einer Reihe von Elementen, die durch Tags markiert sind.
Die grundlegenden Bestandteile eines \ac{XML}-Dokuments sind:

Prolog: Ein optionaler Abschnitt, der die \ac{XML}-Version und die verwendete Zeichencodierung definiert.
In Abbildung \ref{fig:XML Prolog Beispielcode} ist ein häufig genutzter Prolog dargestellt.

\begin{figure}[H]
\centering
\begin{minipage}{0.95\textwidth}
\begin{lstlisting}[language=XML]
<?xml version="1.0" encoding="UTF-8"?>
\end{lstlisting}
\end{minipage}
\caption{XML Prolog Beispielcode}
\label{fig:XML Prolog Beispielcode}
    {Quelle: eigene Darstellung}
\end{figure}

Elemente: Die eigentlichen Daten werden in Elementen gespeichert, die mit einem Start-Tag beginnen und mit einem End-Tag abschließen.
Ein Element kann weitere Elemente enthalten, was zu einer hierarchischen Struktur führt:

\begin{figure}[H]
\centering
\begin{minipage}{0.95\textwidth}
\begin{lstlisting}[language=XML]
<buch>
  <titel>XML-Grundlagen</titel>
  <autor>Max Mustermann</autor>
</buch>
\end{lstlisting}
\end{minipage}
\caption{XML Elemente Beispielcode}
\label{fig:XML Elemente Beispielcode}
    {Quelle: eigene Darstellung}
\end{figure}

Attribute: Jedes Element kann Attribute haben, die zusätzliche Informationen enthalten.
Attribute werden im Start-Tag eines Elements definiert:

\begin{figure}[H]
\centering
\begin{minipage}{0.95\textwidth}
\begin{lstlisting}[language=XML]
<buch genre="Lehrbuch">
  <titel>XML-Grundlagen</titel>
  <autor>Max Mustermann</autor>
</buch>
\end{lstlisting}
\end{minipage}
\caption{XML Attribute Beispielcode}
\label{fig:XML Attribute Beispielcode}
    {Quelle: eigene Darstellung}
\end{figure}

Kommentare: Kommentare werden mit den Tags <!-- und --> eingefügt und dienen der Dokumentation oder dem Hinweis auf bestimmte Teile des Codes,
Sie werden beim Parsen des Dokuments ignoriert.
In Abbildung \ref{fig:XML Kommentare Beispielcode} ist ein Beispiel wie
ein Kommentar genutzt wird.

\begin{figure}[H]
\centering
\begin{minipage}{0.95\textwidth}
\begin{lstlisting}[language=XML]
<!-- Dies ist ein Kommentar -->
\end{lstlisting}
\end{minipage}
\caption{XML Kommentare Beispielcode}
\label{fig:XML Kommentare Beispielcode}
    {Quelle: eigene Darstellung}
\end{figure}

Textinhalt: Zwischen den Start- und End-Tags eines Elements kann Text enthalten sein:

\begin{figure}[H]
\centering
\begin{minipage}{0.95\textwidth}
\begin{lstlisting}[language=XML]
<titel>XML-Grundlagen</titel>
\end{lstlisting}
\end{minipage}
\caption{XML Text Beispielcode}
\label{fig:XML Text Beispielcode}
    {Quelle: eigene Darstellung}
\end{figure}
\cite*{Becher2022}

\color{red}     %Unfertig
\subsubsection{Verarbeiten von XML-Dateien mit Python}

Dieses Kapitel bietet einen Überblick über die grundlegenden Methoden und Techniken, um \ac{XML}-Dateien in verschiedenen Programmiersprachen und mit unterschiedlichen Werkzeugen zu verarbeiten.
Es wird erklärt, wie man \ac{XML}-Dateien parsen, bearbeiten und validieren kann, um sie für unterschiedliche Anwendungen nutzbar zu machen.

Die Analyse von \ac{XML}-Daten

Der erste Schritt, um \ac{XML}-Daten zu verarbeiten, ist das Parsing.
Hierbei wird die \ac{XML}-Datei in ein Programm importiert und in eine Struktur umgewandelt, die das Programm versteht.
Je nach den Anforderungen der Anwendung stehen verschiedene Methoden zur Verfügung, um \ac{XML}-Daten zu parsen.

DOM (Document Object Model)

Ein objektorientiertes Modell, das ein \ac{XML}-Dokument abbildet, ist das DOM.
Es zeigt das gesamte Dokument als Baumstruktur, in der jedes Element, Attribut und der Text als Knoten im Baum angesehen werden.
Eine vollständige in-memory Repräsentation des \ac{XML}-Dokuments ist der Vorteil von DOM, wodurch das Durchsuchen und Bearbeiten des Dokuments einfach möglich ist.
Ein Vorteil des DOM ist, dass es eine einfache Schnittstelle bereitstellt, um \ac{XML}-Daten zu durchsuchen und zu manipulieren.
Es ist jedoch speicherintensiv, weil es das gesamte Dokument in den Arbeitsspeicher lädt, was bei sehr großen \ac{XML}-Dateien problematisch sein kann.

SAX (Simple API for \ac{XML})

Im Gegensatz zu DOM arbeitet SAX als ein ereignisbasierter Parser.
SAX liest die \ac{XML}-Datei sequenziell und erzeugt Ereignisse, wenn es auf bestimmte Tags oder Text stößt.
Es speichert das \ac{XML}-Dokument nicht komplett im Speicher, sondern bearbeitet es während des Lesens.
Aus diesem Grund ist SAX besonders geeignet, um große \ac{XML}-Dateien zu verarbeiten.
Obwohl SAX in Bezug auf den Speicherverbrauch effizient ist, ist es weniger geeignet für die Datenmanipulation, da es keine vollständige Dokumentstruktur erstellt.

ElementTree

ElementTree ist eine einfache und leichtgewichtige Python-Bibliothek, die zur Verarbeitung von \ac{XML} verwendet wird.
Sie bietet eine Baumstruktur, die das \ac{XML}-Dokument effizient abbildet und einfache Methoden zum Navigieren, Bearbeiten und Erstellen von \ac{XML}-Dokumenten bereitstellt.
ElementTree ist einfach zu bedienen und speichereffizient, weshalb es für die meisten Anwendungen geeignet ist, sofern die \ac{XML}-Daten nicht zu groß sind.
Überprüfung von \ac{XML}-Daten
Um sicherzustellen, dass \ac{XML}-Daten einer vordefinierten Struktur entsprechen, ist die Validierung ein entscheidender Schritt.
\ac{XML}-Schemas (XSD) sind dabei ein hilfreiches Werkzeug.
Validierung mittels \ac{XML}-Schema Mit \ac{XML}-Schema wird die Struktur von \ac{XML}-Dokumenten festgelegt.
Es erlaubt die Überprüfung von \ac{XML}-Daten, um zu garantieren, dass sie die richtigen Elemente, Attribute und Datentypen aufweisen.
In Python ermöglicht das lxml-Modul die Überprüfung von \ac{XML}-Daten gegen ein Schema.



\pagebreak