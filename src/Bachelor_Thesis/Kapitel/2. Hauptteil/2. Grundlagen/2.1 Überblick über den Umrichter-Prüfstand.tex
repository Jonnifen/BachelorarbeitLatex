
\subsection{Überblick über den Umrichter-Prüfstand}
In diesem Unterkapitel wird der Umrichter-Teststand, von dem die zu verarbeitenden Datensätze stammen beschrieben,
da dies für das generelle Verständnis der einzulernenden Datensatzstruktur unerlässlich ist.
Die genaue Bezeichnung des Teststandes „USTB DWT Test Bench (XCT0006-1)“ im weiteren als Test-Bench oder Teststand benannt.
Diese Art Teststand wird im allgemein für die End-of-Line Prüfung von unterschiedlichen Umrichtern nach ihrer Herstellung genutzt,
um die Produktqualität und -funktionalität sicherzustellen.\cite*{Main_Manuel_USTB2018}

In dem hier vorliegenden Fall wird der Teststand verwendet, um die aus dem Feld kommenden Umrichter auf ihre weiter Nutzungstauglichkeit zu testen.
Die weiter Nutzungstauglichkeit wird ermittelt, indem die Messwerte mit Mittelwerten, die von mehreren fabrikneuen Umrichtern stammenden verglichen werden.
Diese Messwerte müssen sich in einen vorher definierten Toleranzbereich befinden, um weiter verwenden zu werden.

Die Umrichter werden in der gegebenen Fachliteratur zur Test-Bench als \ac{DUTs} bezeichnet, dies kommt auch in den Testberichten vor,
daher hat der Autor diesen Begriff einzuführen und fortan zu übernehmen.

\subsubsection{Aufbau des Teststandes}
Die Test-Bench besteht aus mehreren Hauptkomponenten:
\begin{itemize}
\item Das Netzteil, welches mit einer 400V Netzspannung versorgt wird, wandelt diese in eine Gleichspannung. Das Netzteil leifert maximal 8kW mit 1200VDC oder 800VDC, welche Werte verwendet werden kann vor Teststart bestimmt werden. In Abbildung 1\ref{fig:1. Aufbau des Teststandes} mit PSU bezeichnet, für „Power Supply Unit“.
\item Das Elektronik-Rack, auf dem Mess- und Control-Komponenten befestigt sind. Hier befinden sich auch der (XCS2100) System-Controller der das ganze System mit dem PC, auf dem die Test-Bench Software läuft, via Ethernet verbindet. In Abbildung \ref{fig:1. Aufbau des Teststandes} mit ER bezeichnet, für „Electronic Rack“.
\item Der Testmatrix-Schrank, in dem die Sammelschienen für den Stromanschluss und den Schützen sitzen. In Abbildung \ref{fig:1. Aufbau des Teststandes} mit TM bezeichnet, für „Test Matrix cabine“.
\item Der Schrank mit dem Kühlungssystem, da die Umrichter während des Betriebes Wassergekühlt werden müssen. In Abbildung \ref{fig:1. Aufbau des Teststandes} mit „Cool1“ bezeichnet.
\item Dem Carrier, auf dem Umrichter befestigt werden. Dieser wird speziell für bestimmte Umrichter konstruiert. In Abbildung \ref{fig:1. Aufbau des Teststandes} mit Carrier1 bezeichnet.

\end{itemize}


\begin{figure}[h]
    \centering
    \includegraphics[width=0.8\textwidth]{Grafiken/Test Cabin.jpg}
    \caption{Aufbau des Teststandes}
    \label{fig:1. Aufbau des Teststandes}
    {Quelle: \cite*[7]{Main_Manuel_USTB2018}}
\end{figure}


Neben dem Hauptkomponenten befinden sich außerhalb des Sicherheitsbereiches, der während des Betriebes nicht betreten werden darf,
ein PC mit einer Software zum Steuern der Testeinrichtung, sowie eine Betriebsanzeige und ein Notaus. \cite*{Main_Manuel_USTB2018}
\color{red}
\subsubsection{Testmodule}

Es gibt verschiedene Testmodule, die auf dem Teststand laufen.
Einige der verschiedenen Funktionen eines DUT können mit dem gleichen Modus eines Testmoduls überprüft werden, indem die entsprechenden Parametersätze ausgewählt werden.
Jeder Test ist autonom und kann mehrmals ausgeführt werden, auch mit unterschiedlichen Parametersätzen.
\\
Im Folgenden wird eine kurze Beschreibung der Funktionen der für diese Arbeit relevanten Testmoduls gegeben.

Driver Consumption Test:

Der Driver-Stromverbrauchstest überprüft den Stromverbrauch des Treibers im Leerlauf und während der PWM-Schaltung.

Pulse Test:

Der Impulstest verfügt über drei Funktionsmodi.
\begin{itemize}
    \item Im Modus „Funktionsschaltung“ (FSW) kann überprüft werden, ob die Halbleiter generell schalten.
    \item Im Modus „Überstromschutz“ (OCP) kann die Überstromüberwachung („weicher Kurzschluss“) überprüft werden.
    \item Im Modus „Dynamischer Kurzschlussschutz“ (DSCP) wird das korrekte Verhalten der Treiberstufe in Bezug auf einen harten Kurzschluss überprüft.
\end{itemize}

Power Test:

Mit diesem Test können zwei verschiedene Funktionen getestet werden:
\begin{itemize}
    \item Einerseits eignet sich der Burn-In-Test (BIT) dazu, die DUTs zyklisch zu betreiben und so reale Betriebszustände zu simulieren.
    \item Andererseits kann durch Überprüfung der Kühltemperatur am Ende des Tests die korrekte Wärmeübertragung der Halbleiter überprüft werden.
\end{itemize}
\\
Während des Übertemperaturschutz-Tests (OTP) wird ein DUT mit reduzierter Kühlung betrieben, bis die maximal zulässige
Kühlkörpertemperatur erreicht ist und die Temperaturschutzschaltung auslöst.

\subsubsection{Testablauf}

Ein Test läuft wie folgt ab:

\begin{enumerate}
\item Die Umrichter werden auf dem Carrier befestigt, es werden meist 3 Umrichter gleichzeitig getastet, Abweichungen je nach Bauform der Umrichter.
Die Reihenfolge der elektrischen Phasen ist bei Draufsicht des Carriers von links nach rechts U-V-W.
\item Anschließen des Kühlkreislaufen, je nach Types Lüft- oder Wasserkühlung.
\item Die \ac{DUTs} werden mit den AC- und DC-Link-Kontakten verbinden.
\item Anbringen von Signalkabeln zwischen DUT und MerkurBox, so wie montieren von VCE-Klemmen an die AC-Kontakte der DUTs.
\item
\item Durchführung des Driver Consumption Testes.
\item Durchführung des Pulse Testes.
\item Durchführung des Driver Power Testes.
\item Nach dem Durchlaufen eines Tests wird automatische ein XML-Datenfile mit den erhobenen Messdaten generiert, die enthalt die Messwerte und alle vorher bestimmten Einstellungen.
\end{enumerate}

