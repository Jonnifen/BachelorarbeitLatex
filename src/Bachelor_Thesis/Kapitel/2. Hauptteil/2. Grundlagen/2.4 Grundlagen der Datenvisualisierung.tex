\subsection{Grundlagen der Datenvisualisierung}
\label{subsec:grundlagen-der-datenvisualisierung}



\subsubsection{Graphen erstellung mit JS}


\textbf{Chart.js}
Eine einfache und beliebte Bibliothek für Standarddiagramme wie Balken, Linien
oder Kreisdiagramme. Schnell eingebunden, gute Standardoptik, ideal für
kleinere Projekte oder schnelle Visualisierungen.

\textbf{D3.js}
Eine sehr mächtige Low-Level-Bibliothek, die mit SVG, Canvas und HTML arbeitet.
Extrem flexibel für individuelle und interaktive Visualisierungen, aber mit
einer steilen Lernkurve.

\textbf{Plotly.js}
Bietet viele interaktive Diagramme (Zoom, Hover, Export als PNG). Unterstützt
auch 3D- und wissenschaftliche Charts. Gut geeignet für Dashboards und
Datenanalyse.

\textbf{ECharts}
Eine leistungsstarke Open-Source-Bibliothek von Apache. Unterstützt viele
Diagrammtypen (inkl. Heatmaps, Maps, Candlesticks) mit Animationen und
Interaktivität. Besonders beliebt für komplexe Enterprise-Web-Apps.

\textbf{Highcharts}
Eine professionelle, kommerzielle Lösung (kostenlos für private Nutzung). Sehr
einfach konfigurierbar, liefert Business-taugliche Diagramme mit vielen
Optionen. Oft in Unternehmen eingesetzt.

\textbf{vis.js}
Spezialisiert auf Netzwerk- und Zeitachsenvisualisierungen. Eignet sich
besonders für Knoten-Graphen, Beziehungen oder Prozessdiagramme mit
Interaktivität.

\textbf{Recharts}
Eine React-spezifische Bibliothek, die auf D3.js basiert.
Bietet eine einfache Komponenten-API für gängige Diagramme.
Ideal, wenn deine Web-App mit React entwickelt ist.


