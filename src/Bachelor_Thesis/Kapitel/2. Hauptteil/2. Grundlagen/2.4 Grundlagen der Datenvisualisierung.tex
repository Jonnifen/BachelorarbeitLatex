\subsection{Grundlagen der Datenvisualisierung}
\label{subsec:grundlagen-der-datenvisualisierung}

In diesem Kapitel der Arbeit wird die Art der Graphen zum visualisieruern der Messdaten erläutert und Anforderungen an die Umsetzung formuliert,
so wie mögliche kurzer Überblick zu Bibliotheken für die Umsetzung durchgeführt.
\subsubsection{Visualisierung von Zeitreihen}

Für das Darstellen der zu visualisierten Testergebnisse sollen nach Claus O. Wilke Liniendiagramme gebutzt werden.
Da die Messwerte der dazustellenden Testmodule anhand des Zeitverlaufes zugeordnet werden, somit haben die Daten eine inhärente Reihenfolge.

Man kann die Messungen nicht einfach vertauschen, ohne die zeitliche Information zu verlieren.
Das bedeutet die Messwerte können nach dieser Zeitreihe geordnet werden und so Vor- und Nachfoger definiert werden, um zusätzlich Information zu erfassen.

Liniendiagramme eignen sich für solche Daten, bei denen die Reihenfolge eine Rolle spielt
zum Beispiel bei Zeitverläufen, Messreihen oder Funktionswerten für eine gute darstellung.
Die Linie im Diagramm verbindet die Punkte in dieser Reihenfolge und zeigt dadurch den Verlauf oder Trend über die Zeit (oder eine andere geordnete Dimension).\cite{Wilke2020Datenvisualisierung}


Wilke, Claus O.. Datenvisualisierung – Grundlagen und Praxis: Wie Sie aussagekräftige Diagramme und Grafiken gestalten (Animals) (S.119). O'Reilly. Kindle-Version.


\subsubsection{Anforderungen an Liniendiagramme nach }

Im folgenden werden Anforderung für das erstellen von gut dargestellten Grafen nach beschrieben. Diese Anforderungen sollen
bei dem erstelle Graphen und der Auswahl der Bibliotheken fürs die Erstellung im späteren Arbeitsverlauf berücksichtigt werden:

\item
Geeignete Skalierung der Achsen

Eine korrekte und sinnvolle Skalierung der Achsen ist entscheidend, um die Daten realistisch darzustellen.
Die Achsen sollten gleichmäßig skaliert und klar beschriftet sein.
Ungleichmäßige Skalierungen oder willkürliche Achsenschnitte können die visuelle Wahrnehmung von Trends verfälschen und sollten vermieden werden.

\item
Klarheit und Lesbarkeit

Liniendiagramme müssen so gestaltet sein, dass die dargestellten Informationen schnell und eindeutig erfassbar sind.
Dazu gehören klar beschriftete Achsentitel, Einheiten und Legenden.
Linien und Beschriftungen sollten gut lesbar sein, ohne sich gegenseitig zu überlagern.
Eine angemessene Linienbreite und Schriftgröße trägt wesentlich zur Verständlichkeit bei.

\item
Konsistente und sparsame Liniengestaltung

Eine konsistente visuelle Gestaltung unterstützt die Vergleichbarkeit zwischen mehreren Linien.
Unterschiedliche Linienarten – etwa durch Farbe oder Strichmuster – sollten nur eingesetzt werden, wenn sie zur Unterscheidung notwendig sind.
Gleiche Bedeutungen müssen stets durch dieselbe visuelle Kodierung wiedergegeben werden. Eine übermäßige Verwendung von Markern oder Symbolen kann die Lesbarkeit beeinträchtigen und sollte vermieden werden.

\item
Farben gezielt und barrierefrei einsetzen
Farben dienen der Unterscheidung und Hervorhebung, sollten jedoch bewusst und zurückhaltend eingesetzt werden.
Die gewählten Farbtöne müssen ausreichend kontrastreich und auch für Personen mit Farbsehschwächen unterscheidbar sein.
Wenn bestimmte Linien besonders hervorgehoben werden sollen, sollte dies dezent und mit klarer inhaltlicher Begründung geschehen, um den Fokus der Betrachtenden zu lenken, ohne die Gesamtwirkung zu stören.

\item
Visueller Überlastung

Überflüssige grafische Elemente wie 3D-Effekte, Schatten, starke Gitterlinien oder dekorative Symbole sollten vermieden werden, da sie die Wahrnehmung der eigentlichen Daten stören.
Ein reduziertes, funktionales Design lenkt den Blick auf die wesentlichen Inhalte und erhöht die Lesbarkeit des Diagramms.

\item
Korrekte Darstellung und Kontext

Ein Liniendiagramm sollte den dargestellten Sachverhalt klar erkennbar machen.
Dazu gehören ein prägnanter Titel, eine verständliche Legende und gegebenenfalls ein Untertitel oder eine erläuternde Beschriftung.
Werden mehrere Diagramme zum Vergleich gezeigt, ist eine einheitliche Achsenskalierung erforderlich, um falsche Eindrücke zu vermeiden und die Vergleichbarkeit sicherzustellen.
\item
Behandlung überlappender Linien

Wenn mehrere Linien in einem Diagramm dargestellt werden, kann es zu Überlappungen kommen, die die Lesbarkeit beeinträchtigen.
In solchen Fällen empfiehlt Wilke den Einsatz von Transparenz, dünneren Linien oder unterschiedlichen Stricharten.
Diese Maßnahmen ermöglichen es, auch bei komplexen Darstellungen den Überblick zu behalten.
\item
Verständliche Hervorhebung wichtiger Daten

Besonders relevante Trends, Ereignisse oder Datenpunkte dürfen hervorgehoben werden, beispielsweise durch eine andere Linienfarbe oder eine beschriftete Markierung.
Solche Hervorhebungen sollten jedoch gezielt und sparsam erfolgen, um den Blick der Betrachtenden zu lenken, ohne die übrigen Daten in den Hintergrund zu drängen.

\subsubsection{Graphen erstellung mit JS}

\subsubsection{Graphen erstellung mit JS}
In diesem Abschnitt werden einige JavaScript-Bibliotheken für das Erstellen von Graphen
in Kurzfassung erläutern, um einen groben Überblick zu erhalten.
Die endgültige Auswahl wird in Kapitel 3 genauer beschrieben.

\textbf{Chart.js}
Eine einfache und beliebte Bibliothek für Standarddiagramme wie Balken, Linien
oder Kreisdiagramme. Schnell eingebunden, gute Standardoptik, ideal für
kleinere Projekte oder schnelle Visualisierungen.

\textbf{D3.js}
Eine sehr mächtige Low-Level-Bibliothek, die mit SVG, Canvas und HTML arbeitet.
Extrem flexibel für individuelle und interaktive Visualisierungen, aber mit
einer steilen Lernkurve.

\textbf{Plotly.js}
Bietet viele interaktive Diagramme (Zoom, Hover, Export als PNG). Unterstützt
auch 3D- und wissenschaftliche Charts. Gut geeignet für Dashboards und
Datenanalyse.

\textbf{ECharts}
Eine leistungsstarke Open-Source-Bibliothek von Apache. Unterstützt viele
Diagrammtypen (z.B. Heatmaps, Maps, Candlesticks) mit Animationen und
Interaktivität. Besonders beliebt für komplexe Enterprise-Web-Applikationen.

\textbf{Highcharts}
Eine professionelle, kommerzielle Lösung (kostenlos für private Nutzung).
Sehr einfach konfigurierbar, liefert Business-taugliche Diagramme mit vielen
Optionen. Oft in Unternehmen eingesetzt.

\textbf{vis.js}
Spezialisiert auf Netzwerk- und Zeitachsenvisualisierungen. Eignet sich
besonders für Knoten-Graphen, Beziehungen oder Prozessdiagramme mit
Interaktivität.

\textbf{Recharts}
Eine React-spezifische Bibliothek, die auf D3.js basiert.
Bietet eine einfache Komponenten-API für gängige Diagramme.
Ideal, wenn eine Web-Applikation mit React entwickelt wird.


