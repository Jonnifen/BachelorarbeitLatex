\subsection{Grundlagen der Datenvisualisierung}
\label{subsec:grundlagen-der-datenvisualisierung}

\subsubsection{Begriff und Zielsetzung}

Unter Datenvisualisierung wird die visuelle Aufbereitung von Daten verstanden, um Muster, Trends, Zusammenhänge
oder Anomalien erkennbar zu machen.
Sie stellt ein zentrales Hilfsmittel dar, um komplexe Informationen effizient zu kommunizieren und kognitive
Verarbeitungsprozesse zu unterstützen[Shneiderman, 1996].
Durch die grafische Darstellung wird es dem Betrachter ermöglicht, große Datenmengen intuitiv zu erfassen,
ohne ausschließlich auf numerische oder textuelle Formen angewiesen zu sein.
Die Zielsetzung der Datenvisualisierung umfasst in der Regel drei Kernaspekte[Ware, 2021]:

\begin{enumerate}

\item
Exploration: Unterstützung bei der Entdeckung neuer Zusammenhänge und Hypothesen.
\item
Analyse: Erleichterung der detaillierten Untersuchung von Strukturen und Abhängigkeiten.
\item
Kommunikation: Vermittlung von Ergebnissen an unterschiedliche Zielgruppen.

\end{enumerate}

\subsubsection{Theoretische Grundlagen}

Die Grundlage wirksamer Datenvisualisierung liegt in der menschlichen Wahrnehmungspsychologie.
Insbesondere die Gestaltungsgesetze (z.B. Gesetz der Nähe, Ähnlichkeit und Kontinuität) spielen eine zentrale Rolle,
da sie bestimmen, wie Informationen visuell gruppiert und interpretiert werden[Wertheimer, 1923].

Zusätzlich beschreibt die Theorie der \textit{kognitiven Belastung} (Cognitive Load Theory),
dass Darstellungen so gestaltet werden sollten, dass sie die Arbeitsgedächtniskapazität nicht überlasten[Sweller, 1988].


Ein weiterer theoretischer Rahmen ist Shneidermans \textit{Visual Information-Seeking Mantra}:



„Overview first, zoom and filter, then details-on-demand.“

Dieses Prinzip betont die Notwendigkeit einer schrittweisen Annäherung an Daten, um vom Gesamtüberblick bis hin zu
spezifischen Details zu gelangen.


\subsubsection{Formen der Datenvisualisierung}

Datenvisualisierungen lassen sich in verschiedene Kategorien einteilen[Few, 2012]:

\begin{itemize}

\item
Explorative Visualisierung: Interaktive Darstellungen, die zur Untersuchung und Hypothesengenerierung dienen.
\item
Explikative Visualisierung: Fokussierte, oft statische Darstellungen, die Ergebnisse gezielt kommunizieren.
\item
Interaktive Dashboards: Kombination mehrerer Visualisierungstypen, häufig für Monitoring und Entscheidungsunterstützung.

\end{itemize}

Je nach Datentyp und Analyseziel kommen unterschiedliche Diagrammformen und Techniken zum Einsatz:

\begin{itemize}

\item
Zeitreihen: Liniendiagramme, Flächendiagramme
\item
Kategorische Daten: Balken- und Säulendiagramme
\item
Zusammenhänge: Streudiagramme, Blasendiagramme
\item
Verteilungen: Histogramme, Boxplots
\item
Hierarchien und Netzwerke: Baumdiagramme, Graphvisualisierungen

\end{itemize}

\subsubsection{Qualitätskriterien}

Eine gute Datenvisualisierung erfüllt folgende Kriterien[Tufte, 2001]:


\begin{itemize}

\item
Klarheit: Vermeidung unnötiger grafischer Elemente („Chartjunk“)
\item
Genauigkeit: Wahrheitsgetreue Darstellung ohne Verzerrung von Skalen oder Proportionen
\item
Effizienz: Schnelle Erfassbarkeit der relevanten Information
\item
Ästhetik: Ansprechende Gestaltung zur Förderung der Akzeptanz
\item
Barrierefreiheit: Berücksichtigung farbsehschwacher Nutzer durch geeignete Farbpaletten

\end{itemize}

\subsubsection{Technologische Aspekte}

Mit dem Fortschritt moderner IT-Systeme haben sich leistungsfähige Werkzeuge und Bibliotheken für die
Datenvisualisierung etabliert.
Beispiele sind Matplotlib und Plotly im Python-Umfeld, D3.js im Webbereich.

In Webapplikationen ermöglichen interaktive Bibliotheken eine nahtlose Einbettung in Benutzeroberflächen,
wodurch sowohl explorative als auch explikative Ziele unterstützt werden.

\pagebreak

\subsubsection{Graphen erstellung mit JS}