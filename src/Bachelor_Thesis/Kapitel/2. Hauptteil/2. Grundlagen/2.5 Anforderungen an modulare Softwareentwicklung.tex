\subsection{Anforderungen an modulare Softwareentwicklung}
\label{subsec:anforderungen-an-modulare-softwareentwicklung}

Die modulare Softwareentwicklung ist ein etabliertes Paradigma, das darauf abzielt, komplexe Softwaresysteme in klar abgegrenzte, wiederverwendbare und unabhängig voneinander entwickelbare Einheiten zu zerlegen.
Ziel ist es, sowohl die Wartbarkeit, Erweiterbarkeit als auch die Qualität der Software zu erhöhen.
Um diese Ziele zu erreichen, müssen spezifische Anforderungen an die Gestaltung und Umsetzung modularer Systeme beachtet werden.


\subsubsection{Klare Modulabgrenzung und Verantwortlichkeiten}

Ein Modul sollte eine klar definierte Aufgabe erfüllen und über eine eindeutig abgegrenzte Funktionalität verfügen.
Diese Trennung wird in der Literatur häufig als \textit{Separation of Concerns} (SoC) bezeichnet 111.
Durch eine eindeutige Abgrenzung lassen sich Abhängigkeiten reduzieren, wodurch Änderungen in einem Modul nur minimale Auswirkungen auf andere Systemkomponenten haben.


\subsubsection{Geringe Kopplung und hohe Kohäsion}

Zwei zentrale Qualitätsmerkmale modularer Systeme sind niedrige Kopplung und hohe Kohäsion 222.


\begin{itemize}

\item
Kohäsion beschreibt, wie eng die Elemente eines Moduls zusammenarbeiten, um eine spezifische Aufgabe zu erfüllen.
\item
Kopplung hingegen beschreibt den Grad der Abhängigkeit zwischen einzelnen Modulen.
Ein hoher Kohäsionsgrad bei gleichzeitig niedriger Kopplung erleichtert sowohl die Wiederverwendung als auch die Wartung.

\end{itemize}

\subsubsection{Einheitliche Schnittstellen}

Module interagieren über klar definierte und stabile Schnittstellen. Diese sollten standardisiert, dokumentiert und möglichst unabhängig von internen Implementierungsdetails sein 333. Eine wohldefinierte API (Application Programming Interface) ermöglicht die parallele Entwicklung mehrerer Module und erleichtert zukünftige Erweiterungen.


\subsubsection{Wiederverwendbarkeit}

Ein wesentliches Ziel der Modularisierung ist die Wiederverwendbarkeit von Modulen in unterschiedlichen Projekten oder Kontexten. Wiederverwendbare Module reduzieren Entwicklungsaufwand, erhöhen die Konsistenz und tragen zur Qualitätssteigerung bei 444. Hierfür ist eine generische und konfigurierbare Implementierung erforderlich.


\subsubsection{Erweiterbarkeit und Anpassungsfähigkeit}

Modulare Software muss so konzipiert sein, dass neue Funktionen ohne weitreichende Änderungen am bestehenden System integriert werden können. Dieses Prinzip wird oft mit dem \textit{Open/Closed Principle} aus den SOLID-Prinzipien beschrieben: „Software entities should be open for extension, but closed for modification“ 555.


\subsubsection{Testbarkeit}

Durch die Abgrenzung von Modulen wird eine gezielte und unabhängige Prüfung einzelner Systemkomponenten möglich. Unit-Tests, Integrationstests und Mocking-Strategien profitieren stark von einer modularen Struktur 666. Testbare Module führen zu einer höheren Softwarequalität und geringeren Fehlerquoten im Betrieb.


\subsubsection{Konsistenz und Standardisierung}

Die Implementierung sollte einheitliche Konventionen in Bezug auf Namensgebung, Dokumentationsstandards und Code-Struktur aufweisen. Konsistenz erleichtert die Zusammenarbeit in Entwicklerteams und minimiert Missverständnisse 777.


\subsubsection{Technologische Unabhängigkeit}

Ein modularer Aufbau sollte es ermöglichen, einzelne Komponenten ohne tiefgreifende Änderungen an andere Plattformen, Technologien oder Bibliotheken anzupassen. Dies reduziert technologische Abhängigkeiten und verlängert die Lebensdauer der Software 888.

