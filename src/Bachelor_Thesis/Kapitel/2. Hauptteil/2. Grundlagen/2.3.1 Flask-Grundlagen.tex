\subsection{Flask-Grundlagen und Funktionsweise}\label{subsec:flask-grundlagen-und-funktionsweise}

In diesem Unterkapitel wird das Webframework Flask, das in der Web-Applikation genutzt wird, behandelt.
Ein grundlegender Überblick wird gegeben und sein wichtiger Funktionsablauf wird erläutert.

\subsubsection{Überblick}

Flask ist ein schlankes Webframework für Python, das auf WSGI (Web Server Gateway Interface) basiert.
Das Web Server Gateway Interface (WSGI) ist eine Schnittstelle, die es Webservern ermöglicht, mit Python-Webanwendungen oder -Frameworks zu kommunizieren.  Der Begriff „leichtgewichtiges“ bedeutet in diesem Zusammenhang, dass der Framework-Kern mit Bedacht minimalistisch entworfen wurde und viele Funktionen nur durch Erweiterungen hinzugefügt werden können.
In der Standardkonfiguration setzt sich Flask aus zwei zentralen Komponenten zusammen: \cite{FlaskDocumentation}
\begin{itemize}

    \item Werkzeug: Ist eine Sammlung von Funktionen zum Erstellen von WSGI-Anwendungen, die HTTP-Requests und -Responses abstrahiert und die Schnittstelle zwischen dem Webserver und der Python-Anwendung ermöglicht.  Außerdem umfasst sie Funktionen zum Aufspüren und Beheben von Fehlern (Debugging), zum Zuordnen von URLs zu Funktionen oder Aktionen in einer Webanwendung (URL-Routing) sowie Hilfsfunktionen für die Arbeit mit HTTP (HTTP-Utilities).

    \item Jinja2: eine Template-Engine, die es ermöglicht, HTML-Dateien mit Platzhaltern, Schleifen, Bedingungen und weiteren Syntaxelementen zu erstellen, um sie dynamisch in der Anwendung zu rendern. Sie erlaubt so die dynamische Erstellung von HTML-Seiten (oder anderen Textdateien), indem Python-Daten in Platzhalter innerhalb einer Vorlage (Template) eingefügt werden.

\end{itemize}


\subsubsection{Grundprinzip der Funktionsweise}

Die beiden obengenannten zentralen Komponenten bilden die Grundlagen für die Funktionsweise von Flask.

Für die Darstellung im Browser werden die Seiten der Applikation serverseitig dynamisch erstellt und an den Client gesendet, indem diese Template-Vorlagen genutzt werden.
Die HTTP-Anfragen (Requests) des Clients werden über die WSGI-Schnittstelle, die Werkzeug bereitstellt, entgegengenommen, von der Anwendung verarbeitet und als HTTP-Antworten (Responses) an den Webserver zurückgeschickt. Hierbei kümmert sich Jinja2 um die dynamische Erstellung der HTML-Seiten, während Werkzeug die Kommunikation zwischen Webserver und Anwendung regelt. Die Auslieferung von dynamisch generierten Webseiten wird durch das Zusammenwirken dieser beiden Komponenten ermöglicht.

Um die Funktionsweise von Flask zu veranschaulichen, wird im Folgenden ein typischer Ablauf einer Webanfrage kurz beschrieben: \cite{FlaskLifecycle,Kennedy2024_FlaskRequests}
\begin{enumerate}

    \item \textbf{Erhalt einer HTTP-Anfrage}\\
Ein Webclient (wie ein Browser) schickt eine Anfrage an den Webserver, der sie über die WSGI-Schnittstelle an die Flask-Anwendung weiterleitet.

    \item \textbf{Routing}\\
Flask untersucht die URL und weist sie einer passenden View-Funktion zu.  Dies erfolgt über sogenannte Routen, die mit dem Dekorator @app.route() festgelegt werden.

    \item \textbf{Bearbeitung der Anfrage}\\
Die Geschäftslogik wird in der zugeordneten Funktion ausgeführt; dazu gehört unter anderem das Auslesen von Parametern, der Zugriff auf eine Datenbank oder die Aufbereitung von Daten.

    \item \textbf{Template Rendering (optional)}\\
Falls die Antwort HTML-basiert ist, wird ein Jinja2-Template mit dynamischen Daten gefüllt und gerendert.

    \item \textbf{Erzeugung der HTTP-Antwort}\\
Flask gibt die fertige Antwort (z. B. HTML, JSON oder Redirect) an den Client zurück.

\end{enumerate}


\subsubsection{Blueprints in Flask}

Flask ermöglicht es, über interne Funktionen und Klassen sogenannte Blueprints zu erstellen und zu nutzen.
Blueprints sind eine strukturierte Möglichkeit, Webanwendungen modular aufzubauen.
Ein Blueprint stellt ein logisches Teilmodul der Anwendung dar und kann eigene Routen, Templates, statische Dateien und Logik enthalten.
Dadurch lassen sich umfangreiche Projekte in klar abgegrenzte Funktionsbereiche unterteilen, was die Übersichtlichkeit und Wartbarkeit der Anwendung erhöht.
Jeder Blueprint wird in einer separaten Datei definiert und anschließend in der Hauptapplikation registriert.
Dieses Vorgehen ermöglicht es, neue Funktionsmodule ohne Eingriff in bestehende Komponenten zu ergänzen.
Somit befördert Blueprint das Unterteilen in modulare Softwarearchitektur, insbesondere die geringe Kopplung und hohe Kohäsion zwischen den Modulen.\cite{FlaskBlueprints}





