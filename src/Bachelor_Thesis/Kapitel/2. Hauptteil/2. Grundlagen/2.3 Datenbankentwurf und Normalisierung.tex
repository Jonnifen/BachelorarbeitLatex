\subsection{Datenbankentwurf und Normalisierung}
\label{subsec:datenbankentwurf-und-normalisierung}

\subsubsection{Einführung in Datenbanken}

Datenbanken sind organisierte Sammlungen von Daten, die elektronisch gespeichert und verwaltet werden.
Ihr primäres Ziel besteht darin, große Datenmengen strukturiert abzulegen, den Zugriff zu optimieren und Datenintegrität sicherzustellen.
Im Gegensatz zu einfachen Dateisystemen bieten Datenbanken Mechanismen zur gleichzeitigen Nutzung durch mehrere Benutzer,
zur Vermeidung redundanter Datenhaltung und zur effizienten Abfrage mittels spezieller Sprachen wie der Structured Query Language (SQL).

Die Entwicklung moderner Informationssysteme erfordert den Einsatz relationaler Datenbanken, dokumentenbasierter Systeme
oder hybrider Modelle, um unterschiedliche Anforderungen an Konsistenz, Flexibilität und Skalierbarkeit zu erfüllen.

\subsubsection{Prozess der Datenbankerstellung}

Die Erstellung einer Datenbank umfasst mehrere aufeinanderfolgende Schritte, die sowohl technische als auch konzeptionelle Aspekte berücksichtigen:


\begin{enumerate}

\item
Anforderungsanalyse


\begin{itemize}

\item
Ermittlung der fachlichen und technischen Anforderungen.
\item
Identifikation der relevanten Datenquellen und Datenformate.
\item
Festlegung von Integritäts- und Sicherheitsanforderungen.

\end{itemize}

\item
Konzeptionelles Datenmodell

\begin{itemize}

\item
Erstellung eines Entity-Relationship-Modells (ERM) zur Abbildung der realen Welt in Entitäten, Attribute und Beziehungen.
\item
Berücksichtigung von Kardinalitäten (1:1, 1:n, n:m).

\end{itemize}

\item
Logisches Datenmodell

\begin{itemize}

\item
Überführung des konzeptionellen Modells in ein relationales Schema.
\item
Definition von Tabellen, Spalten (Attribute), Primärschlüsseln und Fremdschlüsseln.
\item
Festlegung von Datentypen und Constraints (z.B. NOT NULL, UNIQUE, CHECK).

\end{itemize}

\item
Physisches Datenmodell


\begin{itemize}

\item
Umsetzung des logischen Modells in einer konkreten Datenbankmanagementsoftware (z.B. MySQL, PostgreSQL, Microsoft SQL Server).
\item
Optimierung der Speicherstrukturen, Indexierung und Partitionierung.

\end{itemize}

\item
Implementierung und Test

\begin{itemize}

\item
Erstellung der Tabellen und Relationen gemäß Datenbankschema.

\item
Einfügen von Testdaten.

\item
Überprüfung der Funktionalität und Performance.

\end{itemize}

\end{enumerate}


\subsubsection{Datenbankstruktur}

Die Datenbankstruktur bezeichnet den Aufbau und die Anordnung der Datenelemente innerhalb eines Datenbanksystems.
Bei relationalen Datenbanken basiert sie im Wesentlichen auf Tabellen und deren Beziehungen.

Wesentliche Elemente sind:


\begin{itemize}

\item
Tabellen – Grundstruktur, in der Daten in Zeilen (Tupeln) und Spalten (Attributen) gespeichert werden.
\item
Primärschlüssel (Primary Keys) – eindeutige Kennzeichnung eines Datensatzes innerhalb einer Tabelle.
\item
Fremdschlüssel (Foreign Keys) – Verknüpfung zwischen Tabellen, um referenzielle Integrität sicherzustellen.
\item
Indizes – Datenstrukturen zur Beschleunigung von Abfragen.
\item
Views (Sichten) – virtuelle Tabellen, die auf Abfrageergebnissen basieren.
\item
Constraints – Regeln zur Sicherstellung der Datenintegrität (z.B. Wertebereiche, Pflichtfelder).

\end{itemize}

\subsubsection{Normalisierung und Datenintegrität}

Die Normalisierung ist ein methodischer Prozess zur Minimierung von Redundanzen und Anomalien. Sie erfolgt schrittweise in Normalformen (1NF, 2NF, 3NF, BCNF). Jede Normalform beseitigt spezifische Arten von Datenanomalien:

\begin{itemize}

\item
1. Normalform (1NF): Elimination mehrfacher Werte innerhalb einer Zelle; atomare Attribute.
\item
2. Normalform (2NF): Entfernung partieller Abhängigkeiten.
\item
3. Normalform (3NF): Entfernung transitativer Abhängigkeiten.
\end{itemize}

Datenintegrität umfasst die Sicherstellung von Korrektheit, Konsistenz und Vollständigkeit der Daten. Sie wird erreicht durch:

\begin{itemize}
\item
Entity-Integrität (eindeutige Primärschlüssel).
\item
Referentielle Integrität (gültige Fremdschlüsselverweise).

\item
Domänenintegrität (gültige Wertebereiche und Datentypen).
\end{itemize}
