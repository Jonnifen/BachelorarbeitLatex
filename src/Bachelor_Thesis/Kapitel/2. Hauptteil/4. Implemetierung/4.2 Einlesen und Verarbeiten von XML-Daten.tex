\subsection{Einlesen und Verarbeiten von XML-Daten}
\label{subsec:einlesen-und-verarbeiten-von-xml-daten}

Wie bereits in vorherigen Unterkapitel erläutert beginnt das Einlesen und verarbeiten mit der Übergabe des Inhaltes aus der
Upload-Seite der Web-Applikation über die Post-Methode.

Der daraus erhlatenen Daten (\code{xml\_data}) werden in die Funktion \code{upload()} in der route.py für die Upload-Seite

aus den  Nach dem Hochladen wird die XML-Datei vom Modul „xml\_ingest.py“ verarbeitet.
Dieses Modul überprüft die Struktur des Dokuments, extrahiert die relevanten Daten



und überführt sie mittels ORM-Methoden in die relationale Datenbank.
Der XML-Parser basiert auf der Bibliothek lxml, da diese eine effiziente und robuste Verarbeitung von hierarchischen Datenstrukturen ermöglicht.
Der Parser analysiert die Struktur der XML-Dateien, validiert die Inhalte und übergibt sie an das Datenmodell.
Für standardisierte Berichte mit drei Devices under Test (DUTs) funktioniert dieser Prozess zuverlässig.



