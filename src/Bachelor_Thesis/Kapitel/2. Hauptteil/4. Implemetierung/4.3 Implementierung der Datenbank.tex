\subsection{Implementierung der Datenbank}
\label{subsec:implementierung-der-datenbank}

In diesem Unterkapitel wird die Einbindung der Datenbank in die Applikation, so wie die Implemetierung grob erläutert.
Der haupt Fokus leigt hierbei auf der Einbindung und der Methode zum erstellen über Flask-SQLAlchemy.
Die Datenbank wurde mithilfe der Bithiotek Flask-SQLAlchemy erstellt, hierbei wurden die Bithioteken
Flask-Migration für die Verwaltung von Datenbankschemata genutzt.


Implementierung der Datenbank mit Flask-Migrate


funktionsweise

flaskmigrate und  umgesetzt und in die , das als ORM-Schicht
die Kommunikation zwischen Anwendung und MariaDB realisiert.

Grundlage des Datenmodells bildet das in Kapitel 4 entwickelte Entity-Relationship-Modell (ERM).

Die implementierten Tabellen umfassen Entitäten für:

\begin{itemize}

\item
Berichte (report)

\item
Testmodule (module)

\item
Prüfanlagen und Teststände (testbench)

\item
Messdaten (measurement)

\item
Parameter (parameter)

\item
DUTs (dut)

\end{itemize}

Über Fremdschlüsselbeziehungen werden die jeweiligen Abhängigkeiten zwischen den Entitäten abgebildet.
Die Datenbankanbindung erwies sich als stabil; Testeinträge konnten erfolgreich erstellt, abgefragt und gelöscht werden.
Durch die Nutzung des ORM-Modells ist die Implementierung klar strukturiert und leicht wartbar.
Geplant ist eine Erweiterung der Datenbank um Benutzer- und Rollenmodelle, um künftig eine differenzierte Benutzerhierarchie
(z. B. Administratoren, Standardnutzer, Gastzugänge) zu ermöglichen.


