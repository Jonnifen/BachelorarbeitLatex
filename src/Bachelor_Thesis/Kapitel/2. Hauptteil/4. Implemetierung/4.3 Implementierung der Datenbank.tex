\subsection{Implementierung der Datenbank}
\label{subsec:implementierung-der-datenbank}


Die Datenbank wurde mithilfe von Flask-SQLAlchemy umgesetzt, das als ORM-Schicht die Kommunikation zwischen Anwendung und MariaDB realisiert. Grundlage des Datenmodells bildet das in Kapitel 4 entwickelte Entity-Relationship-Modell (ERM).


Die implementierten Tabellen umfassen Entitäten für:


\begin{itemize}

\item
Berichte (report)




\item
Testmodule (module)




\item
Prüfanlagen und Teststände (testbench)




\item
Messdaten (measurement)




\item
Parameter (parameter)




\item
DUTs (dut)




\end{itemize}

Über Fremdschlüsselbeziehungen werden die jeweiligen Abhängigkeiten zwischen den Entitäten abgebildet.

Die Datenbankanbindung erwies sich als stabil; Testeinträge konnten erfolgreich erstellt, abgefragt und gelöscht werden. Durch die Nutzung des ORM-Modells ist die Implementierung klar strukturiert und leicht wartbar.


Geplant ist eine Erweiterung der Datenbank um Benutzer- und Rollenmodelle, um künftig eine differenzierte Benutzerhierarchie (z. B. Administratoren, Standardnutzer, Gastzugänge) zu ermöglichen.


