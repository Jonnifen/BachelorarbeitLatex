\subsection{Entwicklung der Benutzeroberfläche}
\label{subsec:entwicklung-der-benutzeroberflache}
Aufbau der Präsentationsschicht


Die Benutzeroberfläche wurde in HTML, CSS und JavaScript unter Verwendung der Jinja2-Template-Engine realisiert.

Alle relevanten Dateien für die Darstellung der Seiten befinden sich in den Ordnern \code{static} und \code{static}.
Hierbei dient das HTML-Dokument \code{base}




Sie ist in drei Hauptseiten unterteilt:

\begin{enumerate}

\item
Upload-Seite: Hochladen und Einlesen von XML-Dateien,

\item
Berichtstabelle: Übersicht aller eingelesenen Testberichte,

\item
Visualisierungsseite: Darstellung ausgewählter Messdaten in grafischer Form.

\end{enumerate}

Die Navigation erfolgt über eine feste Menüleiste, wodurch eine einheitliche Benutzerführung gewährleistet wird.
Systemmeldungen (z. B. Upload-Erfolg oder Fehlermeldungen) werden direkt auf den jeweiligen Seiten angezeigt.
Die grafische Visualisierung der Messdaten wurde mithilfe der Bibliothek Chart.js umgesetzt.
Sie erlaubt die Darstellung von Linien- und Balkendiagrammen und bildet damit die Grundlage für die geplante interaktive Auswertung.
Die derzeitige Visualisierung erfüllt die funktionalen Anforderungen, soll jedoch im Hinblick auf Darstellungsqualität,
Interaktivität und Layout weiter verbessert werden.


