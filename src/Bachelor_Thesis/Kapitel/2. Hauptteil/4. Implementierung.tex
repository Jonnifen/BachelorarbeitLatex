\newpage
\section{Implementierung}
\label{sec:implementierung}


In diesem Kapitel wird die Umsetzung des im vorherigen beschriebenen Konzepts bis zum Punkt der Abgabe erläutert.
Bei dieser Form der Applikation handelt es sich um einen lauffähigen Prototyp, der bisher nicht verlässlich einsetzbar wäre.
Der Fokus der Implementierung lag bei diesem auf der Realisierung der Kernfunktionen zur Verarbeitung, Speicherung und Darstellung der XML-basierten Testberichte.
Der Prototyp bildet damit die grundsätzlichen Funktionsabläufe der geplanten Web-Applikation ab, stellt jedoch noch kein vollständig ausgereiftes Produkt dar.
Die Mängel, nötigen Änderungen und Erweiterungen bis zu einem nutzbaren Produkt werden in den Kapiteln 5 und 6 behandelt.

Die Implementierung erfolgte schrittweise in iterativen Entwicklungszyklen.
Zunächst wurde das Grundgerüst der Flask-Anwendung erstellt, dann die Einlesefunktion erstellt.
Anschließend die Datenbankanbindung und zuletzt die Benutzeroberfläche mit den Auswertungs- und Visualisierungsfunktionen integriert.
Dabei wurde besonderer Wert auf eine modulare Struktur gelegt, um die Erweiterbarkeit des Systems zu gewährleisten.

Die Beschreibung der Implementierung wird hier an Architekturschichten angelehnt, um eine aufeinander aufbauende Erklärung zu schaffen.

\subsection{Programmstruktur}
\label{subsec:programmstruktur}


Im folgenden Abschnitt wird die genaue Struktur des Programmcodes dargestellt, um mit dieser die in den folgenden
Unterkapiteln zu beschreiben und der Gesamtstruktur besser zuordnen zu können.
Die Programmstruktur folgt dem im Unterkapitel \ref{subsubsec:arch} beschriebenen Strukturmuster.
Eine genaue Beschreibung der Struktur des Programmcodes.
Struktur und der Inhalt der Ordner werden in der folgenden Abbildung \ref{fig: Genaue Struktur des Applikations-Ordners} dargestellt.
Die in der Abbildung vorkommenden Daten entsprechen genau den der Anwendung.


\begin{figure}[H]
    \centering
    \includegraphics[width=1\textwidth]{Grafiken/Min Ordnerstruktur Projekt.png}
    \caption{Genaue Struktur des Applikations-Ordners}
    \label{fig: Genaue Struktur des Applikations-Ordners}
    {Quelle: Eigene Darstellung mit Microsoft Visio}
\end{figure}
\subsection{Entwicklung der Benutzeroberfläche}
\label{subsec:entwicklung-der-benutzeroberflache}
Aufbau der Präsentationsschicht


Die Benutzeroberfläche wurde in HTML, CSS und JavaScript unter Verwendung der Jinja2-Template-Engine realisiert.

Alle relevanten Dateien für die Darstellung der Seiten befinden sich in den Ordnern \code{static} und \code{static}.
Hierbei dient das HTML-Dokument \code{base}




Sie ist in drei Hauptseiten unterteilt:

\begin{enumerate}

\item
Upload-Seite: Hochladen und Einlesen von XML-Dateien,

\item
Berichtstabelle: Übersicht aller eingelesenen Testberichte,

\item
Visualisierungsseite: Darstellung ausgewählter Messdaten in grafischer Form.

\end{enumerate}

Die Navigation erfolgt über eine feste Menüleiste, wodurch eine einheitliche Benutzerführung gewährleistet wird.
Systemmeldungen (z. B. Upload-Erfolg oder Fehlermeldungen) werden direkt auf den jeweiligen Seiten angezeigt.
Die grafische Visualisierung der Messdaten wurde mithilfe der Bibliothek Chart.js umgesetzt.
Sie erlaubt die Darstellung von Linien- und Balkendiagrammen und bildet damit die Grundlage für die geplante interaktive Auswertung.
Die derzeitige Visualisierung erfüllt die funktionalen Anforderungen, soll jedoch im Hinblick auf Darstellungsqualität,
Interaktivität und Layout weiter verbessert werden.



\subsection{Einlesen und Verarbeiten von XML-Daten}
\label{subsec:einlesen-und-verarbeiten-von-xml-daten}


Das Einlesen der Testberichte erfolgt über eine speziell vorgesehene Upload-Seite der Web-Applikation.
Diese Upload-Seite wird verwendet, um die Testberichte einzulesen.
aus den  Nach dem Hochladen wird die XML-Datei vom Modul „xml\_ingest.py“ verarbeitet.
Dieses Modul überprüft die Struktur des Dokuments, extrahiert die relevanten Daten und überführt sie mittels ORM-Methoden in die relationale Datenbank.
Der XML-Parser basiert auf der Bibliothek lxml, da diese eine effiziente und robuste Verarbeitung von hierarchischen Datenstrukturen ermöglicht.
Der Parser analysiert die Struktur der XML-Dateien, validiert die Inhalte und übergibt sie an das Datenmodell.
Für standardisierte Berichte mit drei Devices under Test (DUTs) funktioniert dieser Prozess zuverlässig.




\subsection{Implementierung der Datenbank}
\label{subsec:implementierung-der-datenbank}

In diesem Unterkapitel werden die Einbindung der Datenbank in die Applikation sowie die Implementierung grob erläutert.
Der Hauptfokus liegt hierbei auf der Einbindung und der Methode zum Erstellen über Flask-SQLAlchemy.
Die Datenbank wurde mithilfe der Bibliothek Flask-SQLAlchemy erstellt.
Hierbei wurde die Bibliothek Flask-Migration für die Verwaltung von Datenbankschemata genutzt.

SQLAlchemy stellt \ac{ORM} bereit, mit welchem die relationalen Datenbanktabellen als objektorientierte Python-Klassen dargestellt werden können.
Dadurch konnte die in \ref{subsec:datenbankdesign-und-strukturkonzeption} dargestellte Datenbankstruktur, ohne die aktive Nutzung von SQL-Befehlen, definiert und umgesetzt werden.
Hierbei entspricht jede Klasse einer der Tabellen.
Die Attribute der Klassen bilden die Spalten der Tabellen ab. Die Fremdschlüssel zwischen den Tabellen werden über Relationship-Objekte beschrieben.
Die Klassen werden im Ordner \code{models} definiert.
Zur Veranschaulichung wird der Aufbau an dem Beispiel der Tabelle bzw. der Klasse \code{Report\_Table} aus der Datei \code{report.py} kurz erläutert, siehe Abbildung \ref{fig: Beispiel der Tabelle-Klasse Report-Table}.

\begin{figure}[H]
    \centering
    \includegraphics[width=1\textwidth]{Grafiken/5.4 Class.png}
    \caption{Beispiel der Tabelle-Klasse Report-Table}
    \label{fig: Beispiel der Tabelle-Klasse Report-Table}
    {Quelle: Eigene Darstellung}
\end{figure}



Alle Klassen aus \code{models} sind identisch aufgebaut und folgen demselben Schema.
Alle Klassen besitzen den Import der globalen Instanz \code{db} aus dem Modul \code{extensions}.
Diese Instanz bindet SQLAlchemy in die Flask-Applikation ein und wird beim Starten der Hauptapplikation initialisiert.
Über diese Instanz \code{db} können alle Modelle registriert und Datenbankoperationen wie Abfragen, Einfügungen oder Aktualisierungen ausgeführt werden.
Die Funktion zum Erstellen Hauptapplikation ist in Abbildung \ref{fig: Funktion für das Erstellen der Hauptapplikation} abgebildet, sie liegt in der Datei \code{\_\_init\_\_.py} und wird über \code{run.py} ausgeführt.

\begin{figure}[H]
    \centering
    \includegraphics[width=1\textwidth]{Grafiken/createapp.png}
    \caption{Funktion für das Erstellen der Hauptapplikation}
    \label{fig: Funktion für das Erstellen der Hauptapplikation}
    {Quelle: Eigene Darstellung}
\end{figure}


Die Variable \code{\_\_tablename\_\_} in Abbildung \ref{fig: Beispiel der Tabelle-Klasse Report-Table} legt fest, wie die Tabelle in der Datenbank heißen soll.
In Zeile 5 bis 14 werden die Attribute bzw. die Tabellenspalte definiert.
Am Ende der Klasse wird ein Table-Constraint über \code{\_\_table\_args\_\_} definiert, welches als datenbankseitiger Duplizierungsschutz dient.
Dieser sorgt dafür, dass derselbe XML-Bericht versehentlich mehrfach importiert wird.
Es können nicht zwei Datensätze mit denselben Werten existieren.

Dies dient als zusätzliche Sicherheitsinstanz neben dem selbst erstellten Duplizierungsschutz in \code{ingest\_xml()}.
Die Verbindungen der Datenbank werden über die Klasse Config in der Datei \code{config.py} definiert.
Hierbei wird im Fall, dass die Datenbank nicht vorhanden ist und getestet werden soll, eine SQlLite-Datenbank erstellt und genutzt.
Die genaue Bezeichnung der \code{DATENBANK\_URL} und dem \code{FLASK\_SECRET\_KEY} ist in \code{.env} definiert.
In der folgenden Abbildung \ref{fig:Config-Klasse für Datenbankverbindung} ist die Klasse \code{Config} zum Verbinden mit der Datenbank abgebildet.

\begin{figure}[H]
    \centering
    \includegraphics[width=1\textwidth]{Grafiken/Config.png}
    \caption{Config-Klasse für Datenbankverbindung}
    \label{fig:Config-Klasse für Datenbankverbindung}
    {Quelle: Eigene Darstellung}
\end{figure}

Zur Verwaltung und Versionierung des Datenbankschemas wird Flask-Migrate eingesetzt, welches auf dem Migrationstool Alembic basiert.
Diese Erweiterung erkennt automatisch Änderungen an den ORM-Modellen und erzeugt daraus sogenannte Migrationsskripte. Die Skripte sind im Ordner \code{versions} unter \code{migrations} zu finden.
Diese Skripte enthalten die notwendigen SQL-Befehle, um das Schema der Datenbank an Modellversionen anzupassen.
Durch die Befehle werden die Migrationen generiert und auf die bestehende Datenbank angewendet.
Dadurch ist es möglich, die Struktur der Datenbank schrittweise zu erweitern oder zu verändern, ohne bestehende Daten zu verlieren.
Diese Methode wurde hauptsächlich für die einfachere Entwicklung implementiert und wird nicht zwangsläufig in zukünftigen Versionen weiterverwendet.







\subsection{Technischer Ablauf der Visualisierung}
\label{subsec:technische-details-zur-visualisierung}


Im folgenden Unterkapitel wird der Ablauf des Erstellens der Tabelle für die in der Datenbank enthaltenen Berichte, so wie die
Erstellung der Graphen grob beschrieben und ein Beispiel der Ergebnisse gegeben.

\subsubsection{Erstellen der Berichtstabelle}

Das Erstellen der Berichtstabelle für die Seite „Report-Tabelle“ läuft wie folgt ab:

\begin{enumerate}

    \item Beim Laden der Seite oder beim Bestätigen bzw. Zurücksetzen der Filterbedingungen wird die Funktion \code{report()} der Seite ausgeführt.
    Diese übermittelte die Filterinhalte über die GET-Methode an die Funktion \code{query_reports()} aus der Datei \code{queries.py}.
    Diese Funktion führt über \code{db} einen SELECT-Befehl über die Reportdaten aus, welche die Filterbedingungen erfüllen.
    \item Die ausgewählten Datensätze werden nach Rp\_id geordnet und an die Funktion \code{report()} aus der Route-Datei der Seite zurückgegeben.
    Die Datei \code{route.py} ist für die Veranschaulichung in Abbildung \ref{fig:Code routes.py aus Report-Tabellen-Seite} dargestellt.
    \item Diese Funktion lädt das Template der Seite neu und übergibt die ausgewählten Datensätze.
    \item Durch das erneute Laden wird mittels der HTML-Datei \code{reports.html} und der JavaScript-Datei \code{datatable.js} eine Tabelle mit den
    Ausgewählten Datensätzen erstellt.

\end{enumerate}

\begin{figure}[H]
    \centering
    \includegraphics[width=1\textwidth]{Grafiken/route-reports}
    \caption{Code routes.py aus Report-Tabellen-Seite}
    \label{fig:Code routes.py aus Report-Tabellen-Seite}
    {Quelle: Eigene Darstellung}
\end{figure}

Dieser Ablauf wird jedes Mal, wenn die Seite geladen oder z. B. durch das Ändern der Filterbedingungen refreshed wird, durchgeführt.
Nach dem Laden der Seite für die Report-Tabelle werden erst alle in der Datenbank enthaltenen Berichte ausgegeben, bis Filterargumente festgelegt und bestätigt werden.
Ein Ausschnitt dieser Tabelle ist in Abbildung \ref{fig:Ausschnitt der Report-Tabelle} dargestellt.

\begin{figure}[H]
    \centering
    \includegraphics[width=1\textwidth]{Grafiken/Code route Up}
    \caption{Ausschnitt der Report-Tabelle}
    \label{fig:Ausschnitt der Report-Tabelle}
    {Quelle: Eigene Darstellung}
\end{figure}

\subsubsection{Erstellen der Messwert-Graphen}

Das Erstellen der Graphen ähnelt vom grundlegenden Ablauf her der Seite „Report-Tabelle“.
Der Ablauf für das Erstellen der Graphen der Analyse-Seite läuft wie folgt ab:

\begin{enumerate}

    \item Die Variablen zum Finden des gesuchten \ac{DUTs} können über das Eingeben der in die Suchfelder auf der Analyse-Seite
    oder über Links in der Tabelle auf der Seite der Report-Tabelle eingetragen werden.
    Das Laden dieser Variablen geschieht über die GET-Methode.
    und wird über die Funktion \code{dut\_root()} aus der Route-Datei für die Analyse-Seite durchgeführt.
    \item Die Funktion \code{dut\_root()} führt über \code{db} SELECT-Befehle aus, um die notwendigen Datensätze für die darzustellenden Testmodule (PulsTest und XPowerTest) zu erhalten.
    Hierfür werden Funktionen, unter anderem die Funktion \code{get\_series\_vals\_and\_unit()} aus \code{queries.py}, verwendet, um Datensätze zu suchen.
    Für eine Darstellung der Funktion siehe Abbildung \ref{fig:Beispielcode einer Such-Funktion}.
    \item Die ausgewählten Datensätze werden noch geordnet und am Ende lädt die Funktion die Seite neu.
    \item Beim Neuladen werden die Daten an die HTML-Datei \code{dut\_analyse.html} weitergegeben.
    In dieser werden die geordneten Datensätze mit Hilfe der JavaScript-Dateien \code{dut\_plot.js} und \code{plotly-latest.min.js} verarbeitet und die Graphen für die Analyse-Seite erstellt und ausgegeben.

\end{enumerate}

\begin{figure}[H]
    \centering
    \includegraphics[width=1\textwidth]{Grafiken/get-series.png}
    \caption{Beispielcode einer Such-Funktion}
    \label{fig:Beispielcode einer Such-Funktion}
    {Quelle: Eigene Darstellung}
\end{figure}

Dieser Ablauf wird jedes Mal, wenn die Seite geladen oder z. B. durch das Ändern der Suchbedingungen refreshed wird, durchgeführt.
In der Folge sind zwei auf diese Weise erstellte Graphen dargestellt.
In Abbildung \ref{fig:Beispiel-Graph für PulsTest-Daten} ist ein Graph für die Darstellung der PulsTest-Daten dargestellt.
Die Abbildung \ref{fig:Beispiel-Graph für XPowerTest-Daten} ist ein Graph für die Darstellung der XPowerTest-Daten.

\begin{figure}[H]
    \centering
    \includegraphics[width=1\textwidth]{Grafiken/newplot.png}
    \caption{Beispiel-Graph für PulsTest-Daten}
    \label{fig:Beispiel-Graph für PulsTest-Daten}
    {Quelle: Eigene Darstellung}
\end{figure}

\begin{figure}[H]
    \centering
    \includegraphics[width=1\textwidth]{Grafiken/newplot (1).png}
    \caption{Beispiel-Graph für XPowerTest-Daten}
    \label{fig:Beispiel-Graph für XPowerTest-Daten}
    {Quelle: Eigene Darstellung}
\end{figure}

Die auf diese Weise erstellten Graphen wurden mit den späteren Nutzern überarbeitet, um die fachgerechte Darstellung und Benennung der Graphen zu gewährleisten.


