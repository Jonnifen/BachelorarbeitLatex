\subsection{Testmethode und Durchführung}
\label{subsec:testmethoden-und-durchfuhrung}

Der Prototyp wurde, um seine grundlegenden Funktionen zu testen, mit mehreren möglichst unterschiedlichen XML-Berichten getestet.
Hierfür stand ein Adminreport des Teststandes, der mehrere Testberichte enthält, zur Verfügung.
Es wurden 20 Berichte für Tests ausgewählt.
Es wurde darauf geachtet, zwei Berichte mit der gleichen Materialnummer und somit \ac{DUTs} des gleichen Typs zu verwenden.
Hierbei wurde ein Test mit einem erfolgreichen Testergebnis und ein Test, bei dem ein Fehler aufgetreten ist, zum Testen ausgewählt.
Somit werden 10 DUT-Typen getestet.

Hierbei wurde nachfolgendem Testablauf gefolgt:

\begin{enumerate}

    \item Einlesen des XML-Berichtes, bei Fehler genauere Analyse der Ursache, sonst fortfahren.
    \item Überprüfen der Werte in der Datenbank. Hierbei wurde eine grafische Verwaltungsoberfläche für
    Datenbankmanagementsysteme (HeidiSQL) zu Hilfe genommen, um sich die Dateneinträge genauer erfassen zu können.
    Die Datenbankeinträge des Berichtes werden auf Existenz und Richtigkeit überprüft.
    \item Überprüfen der Ausgabe des Berichtes in der Tabelle auf der Webseite „Report-Tabelle“.
    Der Tabellenbeitrag „Richtigkeit“ überprüft.
    \item Überprüfen der ausgegebenen Graphen und Werte auf der Webseite für das visuelle Ausgeben der Daten.
    Die Graphen und Werte werden auf Existenz und Richtigkeit überprüft.
    \item Eintragen der Ergebnisse in eine Tabelle für einen allgemeinen Überblick.

\end{enumerate}

Die verwendeten XML-Berichte zum Testen wurden im Ordner „Testreports“ im Programmordner gespeichert.
Vor dem Darstellen der Ergebnisse sollte noch angemerkt werden, dass es sich bei allen XML-Berichten um Berichte handelt, die dargestellt werden sollten.
Ausnahme hierbei bildet der DUT-Type mit der ID 115, welcher einer Sondertestform entspricht und nur aus Driver-Consumption-Tests besteht, die hintereinander durchgeführt werden.
Zudem wurde bei DUT-Type mit der ID 110 ein zweiter Test mit positivem Ergebnis verwendet, da keine negativen Tests in dem Adminreport vorhanden waren.
Die Ergebnisse werden so wie die Tabelle im nachfolgenden Unterkapitel dargestellt und erläutert.






