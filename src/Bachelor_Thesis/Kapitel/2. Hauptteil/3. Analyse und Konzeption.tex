%! Author = jbf
%! Date = 31.07.25
\newpage
\section{Analyse und Konzeption}
\label{sec:analyse-und-konzeption}


In diesem Kapitel werden die grundlegenden Anforderungen und konzeptionellen
Überlegungen für die Entwicklung der Webapplikation beschrieben.
Ziel ist es, eine fundierte Basis für die spätere Implementierung zu schaffen, indem sowohl die
funktionalen als auch die nicht-funktionalen Anforderungen analysiert und geeignete technische Konzepte erarbeitet werden.
Dazu werden zunächst die Ausgangssituation und die zu verarbeitenden Datenquellen untersucht.
Die erarbeiteten Konzepte bilden die Grundlage für das Design und die Realisierung der Software im weiteren Verlauf der Arbeit.


\subsection{Definition der Anforderungen}
\label{sec:definition-der-anforderungen-und-ihre-interpretation}

In diesem Abschnitt werden die funktionalen und nicht-funktionalen Anforderungen der zu entwickelnden Web-Applikation beschrieben.
Sie wurden in Abstimmung mit den späteren Nutzerinnen und Nutzern sowie den betreuenden Personen im Unternehmen festgelegt.
Die Anforderungen bilden die Grundlage für den Entwurf und die Umsetzung der Software und dienen dazu, deren Zielsetzung, Funktionsumfang und technische Rahmenbedingungen klar zu definieren.


\subsubsection{Funktionale Anforderungen}\label{subsec:funktionale-anforderungen}

Die funktionalen Anforderungen beschreiben die vorgesehenen Funktionen und Abläufe der Applikation.
Sie definieren, was die Anwendung leisten soll und wie die Nutzerinnen und Nutzer mit ihr interagieren können.

\begin{enumerate}
  \item Navigation und Benutzerführung \\
  Die Applikation soll eine einfache und übersichtliche Navigation ermöglichen.
  Der Wechsel zwischen den Hauptfunktionen erfolgt über eine feststehende Menüleiste im oberen Bereich der Benutzeroberfläche.
  Dadurch können die Bereiche Upload, Berichte und Analyse direkt aufgerufen werden.

  \item Einlesen von XML-Dateien \\
  Das Einlesen der automatisch generierten XML-Berichte erfolgt über eine Upload-Seite.
  Die Anwendung soll sowohl ältere als auch neuere XML-Strukturen verarbeiten können, da der Teststand unterschiedliche Versionen erzeugt.

  \item Datenverarbeitung und Speicherung \\
  Nach dem Upload sollen die XML-Dateien automatisch geparst, validiert und in die Datenbank überführt werden.
  Doppelte Einträge sollen erkannt und vermieden werden.

  \item Anzeige der Berichte \\
  Die gespeicherten Berichte sollen tabellarisch dargestellt werden.
  Jede Zeile repräsentiert einen Testbericht und zeigt die wichtigsten Informationen wie Datum, Seriennummer, Materialnummer und Testergebnis an.

  \item Filter- und Suchfunktionen \\
  Die Berichtstabelle soll über Filteroptionen verfügen, um gezielt nach bestimmten Kriterien
  (z.\,B. Datum, Seriennummer, Testmodul oder Ergebnisstatus) zu suchen.

  \item Grafische Darstellung der Testdaten \\
  Die Anwendung soll die Messergebnisse der einzelnen Module grafisch darstellen.
  Die Diagramme sind nach Modulen sortiert und enthalten eine klare Achsenbeschriftung sowie Legenden.
  Das Layout der Diagramme soll sich an den Graphen des Teststandprogramms orientiert, um den Nutzerinnen und Nutzern die Interpretation zu erleichtern.

  \item Systemmeldungen und Fehlermanagement \\
  Nach erfolgreichen oder fehlerhaften Aktionen (z.\,B. Upload, Datenbankeintrag, Filterabfrage) sollen entsprechende Systemmeldungen angezeigt werden, um den Benutzerstatus transparent zu machen.

  \item Exportfunktionen \\
  Die Anwendung soll die Möglichkeit bieten, ausgewählte Graphen in ein externes Format zu exportieren.
  Dadurch können die Graphen auch außerhalb der Anwendung weiterverarbeitet werden.
\end{enumerate}

\subsubsection{Nicht-funktionale Anforderungen}
\label{subsec:nicht-funktionale-anforderungen}

Die nicht-funktionalen Anforderungen beschreiben die Qualitätsmerkmale der Software.
Sie legen fest, unter welchen Bedingungen die Applikation betrieben werden soll und welche Eigenschaften sie erfüllen muss.

\begin{enumerate}
  \item Plattform und Laufzeitumgebung \\
  Die Web-Applikation wird auf einem unternehmenseigenen Apache2-Server betrieben.
  Sie basiert auf Python und dem Webframework Flask.

  \item Performance \\
  Das System muss auch bei größeren XML-Dateien zuverlässig arbeiten.
  Hierbei wird von XML-Daten mit einer vielzahl von Testmodulen ausgegangen.
  Eine aktive Laufzeitoptimierung ist nicht erforderlich, solange der Arbeitsfluss nicht beeinträchtigt wird.

  \item Usability \\
  Die Benutzeroberfläche soll intuitiv bedienbar sein und eine klare Struktur aufweisen.
  Eingabefehler sollen vermieden und, wenn möglich, automatisch erkannt werden.

  \item Modularität und Wartbarkeit \\
  Die Applikation ist modular aufgebaut, um eine einfache Pflege und spätere Erweiterung zu ermöglichen.
  Änderungen an einzelnen Komponenten sollen keine Anpassungen an anderen Modulen erfordern.

  \item Sicherheit und Datenintegrität \\
  XML-Dateien müssen vor dem Einlesen auf korrekte Struktur und mögliche Sicherheitsrisiken geprüft werden
  (z.\,B. Schutz vor fehlerhaften oder manipulierten Dateien).

  \item Fehler- und Ausnahmebehandlung \\
  Unerwartete Systemfehler sollen abgefangen und im Log gespeichert werden.
  Für Benutzerinnen und Benutzer werden stattdessen verständliche Fehlermeldungen ausgegeben.

  \item Skalierbarkeit \\
  Das System soll bei Bedarf mit minimalem Aufwand um zusätzliche Funktionen, Testmodule oder Datenquellen erweitert werden können.

  \item Kompatibilität \\
  Die Applikation soll auf allen gängigen Browsern lauffähig sein (z.\,B. Chrome, Edge, Firefox).

  \item Dokumentation und Nachvollziehbarkeit \\
  Der Code soll übersichtlich dokumentiert werden.
  Wichtige Abläufe (Upload, Datenverarbeitung, Visualisierung) werden nachvollziehbar beschrieben.
\end{enumerate}

\subsubsection{Anforderungen an die Datenbank}
\label{subsec:anforderungen-an-die-datenbank}

Die Datenbank ist ein zentraler Bestandteil der Anwendung.
Sie speichert alle relevanten Testdaten, Parameter und Informationen aus den XML-Berichten.

\begin{enumerate}

  \item Datenbanksystem \\
  Die Datenbank basiert auf dem relationalen Datenbanksystem MariaDB.

  \item Vollständige Abbildung der XML-Daten \\
  Sie muss in der Lage sein, sämtliche in den XML-Berichten enthaltenen Daten vollständig abzubilden,
  um die Wiederherstellung eines gesamten Testberichts theoretisch zu ermöglichen.

  \item Vermeidung von Datendopplungen \\
  Datendopplungen sollen vermieden werden.
  Häufig vorkommende Informationen (z.\,B. Teststandname, Versionsnummern, DUT-Typen) werden ausgelagert und über Fremdschlüssel referenziert.

  \item Strukturierung von Gerätedaten \\
  Für DUT-Typen und Seriennummern werden separate Informationstabellen angelegt,
  um Redundanzen zu vermeiden und die Datenstruktur klar zu halten.

  \item Erweiterbarkeit der Datenbankstruktur \\
  Die Datenbankstruktur soll so gestaltet sein, dass zukünftige Teststände oder zusätzliche Testmodule problemlos integriert werden können.

  \item Versionierung und Migration \\
  Änderungen an der Datenbankstruktur werden verwaltet,
  um eine konsistente Weiterentwicklung der Datenbank sicherzustellen.

  \item Datenvalidierung und Konsistenzprüfung \\
  Beim Einfügen neuer Datensätze sollen Validierungen sicherstellen,
  dass die Werte logisch und formal korrekt sind (z.\,B. keine Nullwerte bei Pflichtfeldern, korrekte Datentypen).
\end{enumerate}

Die beschriebenen Anforderungen bilden den Rahmen für die Entwicklung der Web-Applikation.
Während die funktionalen Anforderungen die konkreten Aufgaben und Abläufe definieren,
legen die nicht-funktionalen Anforderungen fest, wie die Anwendung qualitativ umgesetzt werden soll.
Zusammen mit den Datenbankanforderungen bilden sie die Grundlage für den folgenden Entwurf der Systemarchitektur und die spätere Implementierung.

\subsection{Grundlegender Ablauf des Programmes}
\label{subsec:grundlegender-ablauf-des-programmes}

Im folgenden Unterkapitel wird der geplante Ablauf der Applikation grundlegend beschrieben.
Hierzu wird ein Ablaufdiagramm zur visuellen Unterstützung verwendet.

Abbildung \ref{fig: Grundlegedes Ablaufdiagramm der Nutzung der Web-Applikation} zeigt den geplanten Ablauf der Datenverarbeitung innerhalb der entwickelten Web-Applikation.
Der Prozess beginnt mit dem Einlesen der XML-Dateien, die die Prüf- und Messdaten enthalten.
Anschließend erfolgt eine Validierung der Datenstruktur und des Formats, um sicherzustellen, dass die XML-Dateien den definierten Spezifikationen entsprechen.
Bei fehlerhaften oder unvollständigen Dateien wird der Prozess abgebrochen.
Sind die Daten gültig, werden sie mit den vorhandenen Datenbankeinträgen abgeglichen.
Abhängig vom Ergebnis werden entweder fehlende Datensätze ergänzt oder neue Einträge vollständig eingefügt.
Dadurch wird sichergestellt, dass die Datenbank stets konsistente und aktuelle Informationen enthält.

Nach dem erfolgreichen Datenimport erfolgt die Auswahl eines Datensatzes, dessen Inhalte grafisch dargestellt werden.
Optional kann der erzeugte Graph als PNG-Datei gespeichert werden.



\begin{figure}[H]
    \centering
    \includegraphics[width=0.95\textwidth]{Grafiken/Ablaufdiagramm}
    \caption{Grundlegedes Ablaufdiagramm der Nutzung der Web-Applikation}
    \label{fig: Grundlegedes Ablaufdiagramm der Nutzung der Web-Applikation}
    {Quelle: Eigene Darstellung mit Microsoft Visio}
\end{figure}
\subsection{Analyse der generierten XML-Berichte und bestehenden Strukturen}
\label{subsec:analyse-der-generierten-xml-berichte-und-bestehenden-strukturen}

trjutkjtr
\subsection{Datenbankdesign und Strukturkonzeption}
\label{subsec:datenbankdesign-und-strukturkonzeption}

In diesem Unterkapitel soll der Verlauf der Erstellung der Datenbank beschrieben und nachvollzogen werden. Es wurde in
den Anforderungen festgelegt, dass MariaDB genutzt werden soll. MariaDB ist ein kostenloses, relationales
Open-Source-Datenbankmanagementsystem, das aus einer Abspaltung von MySQL entwickelt wurde. MySQLs früherer
Hauptentwickler Michael Widenius hat das Projekt ins Leben gerufen. \cite{mariadb.org}

\subsubsection{Betrachtung der Anforderungen}

Für das genaue Interpretieren und Umsetzen der Anforderungen wurde festgelegt, dass die Ersteller/-in der Datenbank erst
ein Konzept nach den festgelegten Anforderungen erarbeiten und umsetzen.
Diese Umsetzung wird dann im Laufe der weiteren Erstellung der Applikation, falls nötig, angepasst.

\subsubsection{Konzeptionelles Datenmodell}

Für das Erstellen der Konzeption des Datenbankmodelles wurde die XML-Struktur analysiert, um die Kenntnisse und Muster
aus Unterkapitel \ref{subsec:analyse-der-generierten-xml-berichte-und-bestehenden-strukturen} abzuleiten.
Aus diesen Erkenntnissen wurde dann ein ER-Modell erstellt, siehe Abbildung \ref{fig: ER-Modell Überlegung der Datenbankstruktur}.
Im Folgenden wird das ER-Modell kurz erläutert.

Durch das Durchführen eines Tests wird ein Bericht erstellt. Dieser Bericht enthält einige Attribute, welche
Testinformationen, darunter die Seriennummern, das Datum und die Uhrzeit des Tests, die Materialnummer (DUT-ID), die
Konfigurations-Version und die Carrier-Bezeichnung enthalten. Zudem besitzt der Bericht ein Unterelement „Testbenchinformationen“, das die Hard- und Softwareversionen des Teststandes enthält. In der Unterentität Testmodelle
befinden sich Versions-, Zeit- und Ergebnisinformationen sowie drei Datensätze, welche die Testparameter und die Messdaten
enthalten. Diese Datensätze können den Phasen des Stromnetzes des Teststandes und somit den DUTs zugeordnet werden.
Zusätzlich zu dem Aufbau wurden hier schon Schlüsselattribute hinzugefügt, die später als Primärschlüssel der
Datenbanktabellen dienen sollen, orange markiert.

\begin{figure}[H]
    \centering
    \includegraphics[width=0.95\textwidth]{Grafiken/Bild von ER-Modell}
    \caption{ER-Modell Überlegung der Datenbankstruktur}
    \label{fig: ER-Modell Überlegung der Datenbankstruktur}
    {Quelle: Eigene Darstellung mit Microsoft Visio}
\end{figure}

Aus diesem ER-Model wurde die Datenbanktabelle und ihre Verbindungen über Fremdschlüssel abgeleitet. In Abbildung \ref{fig: Darstellung der Datenbanktabellen und ihrer Verbindungen}
werden die Datenbanktabellen, ihre genauen Namensbezeichnungen der Tabelle, der Datentyp in der Datenbank und ihr Inhalt sowie ihre Verbindungen abgebildet.

\begin{figure}[H]
    \centering
    \includegraphics[width=0.95\textwidth]{Grafiken/Tabellendiagramm Datenbank}
    \caption{Darstellung der Datenbanktabellen und ihrer Verbindungen}
    \label{fig: Darstellung der Datenbanktabellen und ihrer Verbindungen}
    {Quelle: Eigene Darstellung mit Microsoft Visio}
\end{figure}


Zur persistenten Datenspeicherung wird das ORM-Framework \textit{Flask-SQLAlchemy} verwendet, das auf \textit{SQLAlchemy} basiert.
Es ermöglicht die objektorientierte Abbildung relationaler Datenbanken und vereinfacht den Zugriff auf Daten über Python-Klassen.
Dadurch wird eine saubere Trennung von Anwendungslogik und Datenzugriffsschicht erreicht.



\subsection{Grundkonzept des Benutzeroberflächen-Design}
\label{subsec:grundkonzept-des-benutzeroberflachen-design}

In diesem Abschnitt soll das Grundkonzept der Benutzeroberfläche beschrieben und dargestellt werden.

Die Benutzeroberfläche wird in drei Seiten aufgeteilt, welche sich mit dem Einlesen der Daten, dem
darstellen und auswählen zu visualisierenden Daten und dem Ausgeben das Ergebnis befasst.
Für das Navigieren zwischen den Seiten soll sich auf jeder Seite eine Navigation-Menü im oberen Bereich
der Seite befinden. Zudem sollen alle Seiten einen Bereich für Systemnachrichten wie Bestätigung oder

Fehlermeldungen und ein Label für die Seitenüberschrift besitzen. Diese soll direkt unter den Navigation-
Menü angesiedelt werden, um die Grundstruktur identisch aufzubauen. In den nachfolgenden Text werden
die Konzepte der Benutzeroberflächen und ein grundlegendes Benutzungskonzept der einzelnen Seiten
anhand der Abbildungen \ref{fig: Benutzeroberflächenentwurf der Seite zum Einlesen der XML-Struktur},
\ref{fig: Benutzeroberflächenentwurf der Seite zum Ausgeben der Berichtstabelle} und \ref{fig: Benutzeroberflächenentwurf der Seite zum Ausgeben der Graphen} beschrieben.

Für das Einlesen der XML-Daten soll ein Eingabebox genutzt werden, in die die XML-Struktur kopiert wird.
Die Eingaben soll über einen Button unter der Eingabebox bestätigt werden. Bei einem erfolgreichen
Einlesen und einfügen in die Datenbank soll eine Bestätigungsnachricht im Bereich für
Systemnachrichten erscheinen. Bei einem Fehler soll in diesem Bereich eine Fehlermeldung mit
möglicher Lösung erscheinen. Der Aufbau dieser Seite ist in der Abbildung \ref{fig: Benutzeroberflächenentwurf der Seite zum Einlesen der XML-Struktur} dargestellt.

\begin{figure}[H]
    \centering
    \includegraphics[width=0.95\textwidth]{Grafiken/Overlay_Einleseseite}
    \caption{Benutzeroberflächenentwurf der Seite zum Einlesen der XML-Struktur}
    \label{fig: Benutzeroberflächenentwurf der Seite zum Einlesen der XML-Struktur}
    {Quelle: Eigene Darstellung mit Microsoft Visio}
\end{figure}

Die Seite für das Darstellen und Auswählen zu visualisierenden Daten soll im oberen Abschnitt der Seite
ein Bereich für Filtereinstellungen besitzen, mit möglichen Filteroption wie Datum oder Seriennummer.
Neben dem Filtereinstellung soll zwei Button für das Bestätigen und Zurücksetzen der Filtereinstellungen
sein. Bei einem Fehler bei der Filtereingaben soll in Bereich für Systemnachrichten eine Fehlermeldung
mit möglicher Lösung erscheinen.
Im unteren Bereich der Seite soll sich eine Tabelle, welche die eingelesenen Berichte in der Datenbank
anzeigt. Diese soll durch die Filtereinstellungen angepasst werden. Hierbei muss auf eine Anzeige
Möglichkeit für länge Tabellenstrukturen berücksichtigt werden, um die Benutzung effizient zu halten. Das
Grundkonzept ist in Abbildung \ref{fig: Benutzeroberflächenentwurf der Seite zum Ausgeben der Berichtstabelle} dargestellt.

\begin{figure}[H]
    \centering
    \includegraphics[width=0.95\textwidth]{Grafiken/Overlay_Tabellenseite}
    \caption{Benutzeroberflächenentwurf der Seite zum Ausgeben der Berichtstabelle}
    \label{fig: Benutzeroberflächenentwurf der Seite zum Ausgeben der Berichtstabelle}
    {Quelle: Eigene Darstellung mit Microsoft Visio}
\end{figure}

Für das Ausgeben der ausgewallten Berichtsdaten soll im oberen Abschnitt der Seite eine Instanz
eingefügt werden, die die Information zu dem Ausgewählten Berichtsdaten angezeigt werden.
Im unteren Bereich soll ein Label für die Seriennummer der Ausgewählten Berichtsdaten eingefügt
werden. Darunter werden Enthalten den Modulen mit Labeln benannt und unter den Modulname werden
die Graphen und Werte für die visuelle Darstellung der Testdaten eingefügt. Die Anzahl der Darstellungen
hängt von den in den Bericht Daten enthaltenden Modulen ab. Die Anzahl und der Aufbau des Unteren
Bereiches ist variabel. Der grundlegende Aufbau dieser Seite ist in der Abbildung \ref{fig: Benutzeroberflächenentwurf der Seite zum Ausgeben der Graphen} dargestellt.

\begin{figure}[H]
    \centering
    \includegraphics[width=0.95\textwidth]{Grafiken/Overlay_Ausgabeseite}
    \caption{Benutzeroberflächenentwurf der Seite zum Ausgeben der Graphen}
    \label{fig: Benutzeroberflächenentwurf der Seite zum Ausgeben der Graphen}
    {Quelle: Eigene Darstellung mit Microsoft Visio}
\end{figure}
\subsection{Entwurf der Applikationsarchitektur}
\label{subsec:entwurf-der-applikationsarchitektur}
