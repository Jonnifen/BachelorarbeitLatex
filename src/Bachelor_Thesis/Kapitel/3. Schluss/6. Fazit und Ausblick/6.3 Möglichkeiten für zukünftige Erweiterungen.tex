\subsection{Ausblick und möglichkeiten für zukünftige Erweiterungen}
\label{subsec:moglichkeiten-fur-zukunftige-erweiterungen}

Diese Arbeit bietet eine vielversprechende Grundlage für zukünftige Arbeiten und Erweiterungen.
Der aktuelle Prototyp erfüllt bereits erfolgreich die essenziellen Funktionen zur Verarbeitung, Speicherung und Anzeige von XML-Daten, hat jedoch noch Schwächen in den Bereichen Stabilität, Benutzerführung und Visualisierung.

Der nächste wesentliche Schritt ist die Behebung der durch das Testen bekannten Mängel und die Entwicklung von automatisierten Tests,
welche alle bekannten Varianten der XML-Berichte abdecken und die Datenverarbeitung auf logische Konsistenz prüfen, um somit das Einlesen und Verarbeiten für alle Testberichte zu gewährleisten.
Zudem sollte die Stabilität und Wartbarkeit des Applikationscodes optimiert werden.

Nach dem Beheben der grundlegenden Mängel sollte die grafische Analyse optimiert werden, beispielsweise durch Vergleichsansichten und statistische Auswertungen.
So wie die Integration einer Benutzerverwaltung mit unterschiedlichen Rollen (z. B. Administrator, Prüfer, Techniker), um den Zugriff auf Daten zu steuern.

Mittelfristig soll die Applikation weiterentwickelt werden, um vollständige Kundenberichte zu erstellen, die sowohl intern als auch extern verwendet werden, und um die alten Speichermethoden vollständig zu ersetzen.

Sobald die Datenbank als Bericht umfasst, soll langfristig auf eine Entwicklung der Applikation zu einer Analyseplattform für die Erkennung der Alterungsprozesse der \ac{DUTs} hingeführt werden.
Hierbei könnte auch Machine Learning zur Erkennung von Fehlerbildern einen möglichen Entwicklungsansatz darstellen.

Insgesamt zeigt der aktuelle Prototyp, dass das zugrunde liegende Konzept tragfähig ist und ein erhebliches Potenzial für die zukünftige Nutzung im produktiven Umfeld bietet.
Hierbei sind die mittelfristigen Ziele mit diesem Ansatz mit großer Sicherheit umsetzbar.
Für die langfristigen Ziele der Applikation müsste zu einem fortgeschrittenen Zeitpunkt eine Evaluierung durchgeführt werden.





