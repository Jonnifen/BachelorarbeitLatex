\subsection{Kritische Bewertung}
\label{subsec:kritische-bewertung}

Bewertend kann festgehalten werden, dass die Applikation bzw. der entwickelte Prototyp der Anwendung alle wesentlichen Funktionen umfasst und die vorgesehenen Aufgaben zuverlässig ausführt.
Lediglich das Einlesen einiger spezieller Berichtsstrukturen bereitet noch Schwierigkeiten für die Applikation.

In Rücksprachen mit den späteren Nutzerinnen und Nutzern zeigte sich, dass die Applikation insgesamt als hilfreich, intuitiv und praxisnah wahrgenommen wird.
Das Feedback bestätigte insbesondere die einfache Bedienbarkeit und die verständliche Strukturierung der Benutzeroberfläche.
Dennoch konnten im Zuge der Rücksprachen verschiedene Aspekte aufgezeigt werden, die im weiteren Verlauf gezielt optimiert werden sollten.

So bietet die derzeitige grafische Darstellung der Prüfergebnisse noch keine umfassende Bewertung oder vergleichende Analyse zwischen verschiedenen Testberichten.
Es wird sich derzeit auf einzelne Berichtsteile in der Analyse beschränkt, womit die Einordnung in einen Gesamtkontext derzeit nicht hinreichend gegeben ist.
Nach dem derzeitigen Stand kann die Applikation bereits zur Analyse für das Fachpersonal genutzt werden, erfordert jedoch noch Anpassungen und Erweiterungen, um in der geplanten Weise im betrieblichen Umfeld eingesetzt werden zu können.

Zudem sollten intensive Tests durchgeführt werden, welche alle bisher bekannten Typen und Fehlervarianten des Teststandes bestmöglich abdecken, um die Applikation weiter zu validieren und zu optimieren.
Die bisherige Testmethode wurde zum Erkennen grundlegender Mängel genutzt und kann einen Erfolg des Ansatzes nachweisen.
Es sollten jedoch automatisierte und gründliche Tests für das Überprüfen der Applikation erstellt werden, um eine noch präziser Mängel in der Applikation zu erkennen.

Bezüglich des Codes der Applikation sind insbesondere die Dokumentation und der Aufbau der Funktionen noch zu optimieren, um die Lesbarkeit des Codes nachhaltig zu verbessern.
Die Applikation besitzt zurzeit einige Funktionen, welche ähnliche Aufgabenbereiche abdecken und zu einer generelleren und somit flexibleren Funktion zusammengeführt werden sollten.

Insgesamt ist der Ansatz des Konzeptes als geeignet einzuschätzen und kann mit einigen gezielten Anpassungen und Erweiterungen zu einer verlässlichen und ausgereiften Applikation führen.
Insbesondere die Wahl der genutzten Bibliotheken und Methoden sowie die gewählte Systemarchitektur haben sich als zweckmäßig und tragfähig für die Erstellung der
vorgesehenen Applikation herausgestellt.












