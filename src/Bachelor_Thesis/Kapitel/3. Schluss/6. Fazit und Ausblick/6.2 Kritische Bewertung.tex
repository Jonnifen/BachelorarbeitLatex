\subsection{Kritische Bewertung}
\label{subsec:kritische-bewertung}

Obwohl die entwickelte Applikation bzw. der Prototyp die Kernfunktionen weitestgehend erfüllt und es positives Feedback der späteren Nutzer gegeben hat,
zeigt die Konzeption und die derzeitige Umsetzung noch große Schwächen auf.
Insbesondere eine aussagekräftige Visualisierung der Berichtsdaten in Bezug auf die Aussage der Testergebnisse ist mangelhaft.
Die Darstellung sagt wenig über die Einordnung der Berichte aus.
Es wird sich nur auf einzelne Berichtsteile in der Analyse beschränkt, womit die Einordnung in einen Gesamtkontext nicht ausreichend dargestellt ist.

Nach dem derzeitigen Konzept kann die Applikation nur zur Analyse für das Fachpersonal eingesetzt werden und benötigt noch viele Anpassungen und Erweiterungen, um in der geplanten Umgebung eingesetzt zu werden.
Hierbei müssen zudem intensive Tests durchgeführt werden, welche alle bisher bekannten Typen und Fehlervarianten des Teststandes bestmöglich abdecken.
Die bisherigen Tests waren nur zum Erkennen grundlegender Mängel zu nutzen und können nur einen grundlegenden Erfolg des Ansatzes belegen.
Es müssen automatisierte und gründliche Tests für das Prüfen der Applikation erstellt werden, um eine genauere Einschätzung zu gewinnen.

Ein weiterer Schwachpunkt ist die Struktur und Lesbarkeit des Applikationscodes. Diese ist kaum vorhanden.
Die Applikation besitzt zu viele ansatzweise identische Funktionen, welche ähnliche Aufgabenbereiche haben und zu einer generelleren und somit flexibleren Funktion zusammengefasst werden sollten.

Dennoch sind die Grundstruktur und Ansätze des Konzeptes vielversprechend und können mit einigen geringeren Anpassungen und Erweiterungen zu einer geeigneten Applikation führen.
Insbesondere die Wahl der genutzten Bibliotheken und Methoden sowie die gewählte Systemarchitektur haben sich als geeignet für die Erstellung der gewünschten Applikation herausgestellt.

Der derzeitige Prototyp bzw. Applikationsstand stellt hierbei nur einen hastigen Ansatz dar, der aufgrund der allgemein knapp bemessenen Zeit und des Umfangs des Projektes nicht alle gewünschten Ziele
zur Gänze erfüllen konnten.
Dennoch bestätigt der aktuelle Stand die prinzipielle Umsetzbarkeit und Leistungsfähigkeit des gewählten Konzeptes.








