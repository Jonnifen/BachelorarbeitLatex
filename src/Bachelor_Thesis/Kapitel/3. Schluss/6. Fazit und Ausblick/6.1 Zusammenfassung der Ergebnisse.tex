\subsection{Zusammenfassung der Ergebnisse}
\label{subsec:zusammenfassung-der-ergebnisse}

Ziel dieser Arbeit war die Entwicklung einer modular aufgebauten, serverbasierten Web-Applikation zur Verarbeitung,
Speicherung und Visualisierung von XML-Daten, die aus den Prüfverfahren eines Umrichter-Teststands stammen.
Die Anwendung sollte die automatisch erzeugten XML-Berichte einlesen,
die enthaltenen Messdaten in einer Datenbank speichern und anschließend grafisch darstellen.

Die Arbeit umfasste die Grundlagen zur Erstellung, Beschreibung und Bewertung der Applikation, die fundamentale Konzeption, die Implementierung sowie die Testung eines funktionalen Prototyps.

Bei der Konzeptentwicklung und -umsetzung wurde insbesondere auf die vorher festgelegten Anforderungen sowie die während des Erstellungsprozesses geforderten Änderungsvorschläge durch das Unternehmen geachtet.
Zudem wurde besonderer Wert auf eine modulare, erweiterbare und wartbare Architektur gelegt, um schnelle Änderungen durchzuführen.
Durch den Einsatz von Flask als Webframework und Flask-Blueprint konnte eine klare Trennung zwischen Präsentations-, Logik- und Datenhaltungsschicht umgesetzt werden.
Das ORM-Framework Flask-SQLAlchemy wurde hierbei für die Datenbankerstellung und -anbindung genutzt, wodurch die Verarbeitung und Verwaltung der XML-Strukturen effizient und strukturiert realisiert werden konnte.

Die entwickelte Applikation bzw. der Applikations-Prototyp ist in der Lage, einen Großteil der vom Teststand erzeugten XML-Berichte automatisiert zu verarbeiten, redundante Einträge zu vermeiden und die Daten konsistent in der Datenbank abzulegen.
Über die Benutzeroberfläche können vorhandene Berichte durchsucht, gefiltert und ausgewählte Datensätze visualisiert werden.
Damit wurde das Hauptziel dieser Arbeit, die automatisierte Datenaufbereitung und visuelle Darstellung von Prüfdaten, weitestgehend erreicht.
Hierfür wird eine Unregelmäßigkeit in der Berichtstruktur zugrunde gelegt, die dem Entwickler entgangen ist und durch bestimmte Bauformen der zu testenden Einrichtung ausgelöst wird.



