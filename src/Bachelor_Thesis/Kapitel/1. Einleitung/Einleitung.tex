%! Author = jbf
%! Date = 25.07.25

% Document


\newpage
\section{Einleitung}
\label{sec:einleitung}


„Data is the new oil. Like oil, data is valuable, but if unrefined it cannot really be used.“
Clive Humby (2006)

Die Aufbereitung von Daten, sowohl in ihrer Struktur als auch in ihrer Darstellung, ist entscheidend, um technische Prozesse zu verstehen und effizient mit großen Datenmengen arbeiten zu können.
Das Erfassen, Aufbereiten und Interpretieren von Messdaten ist heute aus der Industrie und Wissenschaft nicht mehr wegzudenken.
In nahezu allen technischen Bereichen werden Daten erhoben, sei es in der Forschung, bei Qualitätsprüfungen oder in industriellen Testverfahren.
Damit diese Daten nutzbar sind, müssen sie verständlich aufbereitet und übersichtlich dargestellt werden.

Die Verarbeitung und Visualisierung großer Datenmengen erfolgt heutzutage meist mithilfe spezialisierter Softwarelösungen.
Viele dieser Systeme sind jedoch allgemein gehalten und nur begrenzt auf spezifische technische Anwendungen anpassbar.
Das kann sich negativ auf Effizienz, Benutzerfreundlichkeit und Aussagekraft der Ergebnisse auswirken.
Besonders in Bereichen, in denen die Datensätze komplex und individuell strukturiert sind, wie bei der Prüfung von Leistungselektronik aus Windenergieanlagen, besteht Bedarf an speziell angepassten Softwarelösungen.

Ziel dieser Arbeit ist es, eine Web-Applikation zu entwickeln, die automatisch erzeugte XML-Berichte eines Umrichter-Teststandes einliest, die enthaltenen Mess- und Gerätedaten in einer Datenbank speichert und diese anschließend grafisch aufbereitet.
Auf diese Weise sollen die Daten besser ausgewertet und sowohl für interne Analysen als auch für externe Berichte genutzt werden können.
Dabei liegt der Fokus auf einer übersichtlichen Benutzeroberfläche, einer modularen Erweiterbarkeit und einer einfachen Integration in die bestehende Systemlandschaft des Unternehmens.

Die Arbeit entstand in Zusammenarbeit mit einem Serviceanbieter für die Instandhaltung von Windenergieanlagen.
Der zugrunde liegende Teststand wird zur Überprüfung von Umrichtern eingesetzt, die aus bestehenden Anlagen stammen.
Mithilfe spezieller Prüfverfahren wird bestimmt, ob diese Geräte nach der Instandsetzung wiederverwendet werden können.
Die dabei entstehenden XML-Dateien enthalten sämtliche Messwerte, Parameter und Prüfstandsinformationen und bilden die Grundlage für die zu entwickelnde Software.

Für die Umsetzung wurde ein iteratives Vorgehen gewählt.
Das bedeutet, dass die Anwendung schrittweise entwickelt und nach jedem Zwischenschritt getestet und verbessert wird.
Dieses Vorgehen orientiert sich an Prinzipien agiler Softwareentwicklung, um flexibel auf mögliche Anpassungen reagieren zu können.
So entsteht eine funktionierende Anwendung, die sich im Laufe der Entwicklung immer weiter verfeinern lässt.

Die Arbeit gliedert sich in sechs Kapitel.
Nach dieser Einleitung werden in Kapitel 2 die theoretischen und technologischen Grundlagen behandelt, die für die Entwicklung der Anwendung relevant sind.
In Kapitel 3 folgen die Analyse der vorhandenen Strukturen und der Entwurf der Systemarchitektur, sowie eine Beschreibung der funktionalen und nicht-funktionalen Anforderungen beschrieben.
Kapitel 4 beschreibt die Implementierung der Applikation, während Kapitel 5 die Integration in die bestehende Systemumgebung und die Testdurchführung erläutert.
Kapitel 6 fasst die Ergebnisse zusammen und gibt einen Ausblick auf mögliche Erweiterungen.
















