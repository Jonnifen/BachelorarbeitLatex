%! Author = jbf
%! Date = 25.07.25

% Document


\newpage
\section{Einleitung}
\label{sec:einleitung}
Das Aufbereiten von Daten, sowohl in Bezug auf die Struktur, in der die Datensätze gespeichert werden,
als auch in der visuellen Darstellung einzelner Datenreihen, ist für das
Verstehen der dahinterliegenden Prozesse und der effizienten Arbeit mit ihnen unerlässlich.
Das Erfassen von Messdaten und ihre Aufbereitung, so wie ihre Interpretation
ist aus der modernen Welt nicht mehr wegzudenken.
In allen Bereichen der Wissenschaft werden Messdaten erhoben, egal ob bei medizinischen Studien,
Wetterdaten oder in der Industrie erhobenen Testdaten, Sie müssen alle
verständlich aufbereitet werden.

Die Speicherung,Aufbereitung und Visualisierung von größeren Datenmengen wir heutzutage meisten mit Softwaretools durchgeführt.
Aus diesem Grund gibt es eine Vielzahl von Anbietern die
Lizenzen für solchen Softwaretools vertreiben.
Diese Tools sind oft aber allgemein gehalten und biete daher für fachspezifische Bereiche nicht den
passenden Aufbau und Funktionen, welches sich Effiziente, Umgänglichkeit und
vor alledem in der Genauigkeit widerspiegelt.

Das Ziel dieser Arbeit ist es, eine Web-Applikation zu entwickeln, die als solch ein Softwaretool
fungieren soll, welches spezifisch für einen Anwendungsbereich zugeschnitten ist.
Die Applikation soll XML-Datensätze, die aus einem Umrichter-Teststand
stammen strukturiert Speicher und visuell Darstellen kommen.
Diese visuellen Daten sollen sowohl firmenintern zur Analyse als auch als in einem
Kundenbericht nach außen weitergegeben werden, daher soll die Visualisierung sowohl
einfach verständlich also auch professionell gehalten sein, hierbei soll
möglichst auf eine intuitive und simple Nutzeroberfläche geachtet werden.
Weiterhin soll die Applikation gut erweiterbar sein, um zukünftige Funktionen
unkompliziert in die Software einzubetten.
Zudem soll eine möglichst einfache Implementierung in die bestehende Softwareumgebung der Firma in dessen Rahmen
diese Arbeite durchgeführt wird gewährleistet sein.

Um das gesetzt Ziel und die Anforderungen im Zeitrahmen dieser Arbeit bestmöglich zu erfüllen, wird nach
dem Erlangen und Ausformulieren der theoretischen Grundlagen der Erstellung für
den eigentlichen Entwicklungsprozess der Web-Applikation eine inkrementelle und
iterative Vorgehensweise vorgesehen.
Diese soll für Flexibilität und gutes Risikomanagement in Bezug auf die Abgabe sorgen, da selbst bei Komplikation
eine funktionelle Version zur Abgabe bereitsteht.
Dieser Ansatz zur Entwicklung der Software wird sich nur geringfügig auf den Aufbaue der
wissenschaftlichen Arbeit auswirken, sie wird nur in den theoretischen
Grundlagen und im Fazit behandelt werden.

Der allgemeine Aufbau des Hauptteils der Arbeit beginnt mit den Kapitel \ref{sec:grundlagen}. Grundlagen,
die sich mit den theoretischen Hintergründen zur
Entwicklung der Applikation, sowie mit den einzelnen zu implementierenden Funktionen beschäftigt.
Gefolgt von Kapitel \ref{sec:analyse-und-konzeption}. Analyse und Konzeption, welches sich mit dem bestehenden System
und den einzufügenden Strukturen befasst.
Das dritte Kapitel \ref{sec:implementierung}. Implemetierung beschreibt die Entwicklung der Web-Applikation.
Kapitel \ref{sec:integration-und-test}. Integration und Test beschreibt die Imtegration in das bestehende System und
das Testen der Funktionalität.
Abschließend Kapitel \ref{sec:fazit-und-ausblick}. Fazit und Ausblick werden die Ergebnisse zusammengefasst
und ein Ausblick auf mögliche Erweiterungen gegeben.





