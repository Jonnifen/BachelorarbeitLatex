%! Author = jbf
%! Date = 25.07.25

% Document


\newpage
\section{Einleitung}
\label{Einleitung}
In der modernen Antriebstechnik spielen Umrichter eine zentrale Rolle bei der Steuerung und Regelung elektrischer Maschinen.
Um ihre Leistungsfähigkeit und Zuverlässigkeit zu gewährleisten, werden sie in Testbenches unter verschiedenen Betriebsbedingungen geprüft.
Dabei entstehen große Mengen an Messdaten, die oft in XML-Formaten vorliegen.
Die manuelle Auswertung dieser Rohdaten ist zeitaufwendig, fehleranfällig und erschwert die schnelle Identifikation relevanter Muster oder Anomalien.
Besonders bei komplexen Testläufen kann der fehlende direkte Zugriff auf übersichtlich aufbereitete Ergebnisse den Entwicklungsprozess verlangsamen.
Ziel dieser Arbeit ist es, eine Web-Applikation zu entwickeln, die XML-Daten aus Umrichter-Testbenches automatisiert einliest, in einer Datenbank speichert und interaktiv visualisiert.
Dadurch soll die Auswertung vereinfacht, die Datenanalyse beschleunigt und die Entscheidungsfindung im Entwicklungsprozess unterstützt werden.
Die Lösung wird mit modernen Webtechnologien umgesetzt und legt den Fokus auf eine effiziente Datenverarbeitung, flexible Filtermöglichkeiten sowie eine intuitive Benutzeroberfläche.




