%! Author = jbf
%! Date = 25.07.25

% Document


\newpage
\section{Einleitung}
\label{sec:einleitung}






„Data is the new oil. Like oil, data is valuable, but if
unrefined it cannot really be used.“

– Clive Humby (2006).

Die Aufbereitung von Daten, sowohl hinsichtlich ihrer
Speicherstruktur als auch ihrer visuellen Darstellung, ist entscheidend, um die
zugrunde liegenden Prozesse zu verstehen und effizient mit den Daten arbeiten
zu können. Das Erfassen, Aufbereiten und Interpretieren von Messdaten ist aus
der modernen Welt nicht mehr wegzudenken. In nahezu allen Bereichen der
Wissenschaft werden Messdaten erhoben, sei es in medizinischen Studien, bei
Wetteranalysen oder in industriellen Testverfahren. Damit diese Daten nutzbar
sind, müssen sie verständlich aufbereitet werden.



Die Speicherung, Verarbeitung und Visualisierung großer
Datenmengen erfolgt heutzutage überwiegend mithilfe spezialisierter
Softwaretools. Entsprechend existiert eine Vielzahl kommerzieller Anbieter, die
Lizenzen für solche Lösungen vertreiben. Diese Tools sind jedoch meist
allgemein gehalten und bieten für fachspezifische Anwendungen nicht immer den
passenden Aufbau oder die erforderlichen Funktionen. Dies wirkt sich
insbesondere auf Effizienz, Benutzerfreundlichkeit und Genauigkeit aus. Ein
Beispiel für die praktische Relevanz der Datenaufbereitung ist die Anwendung im
Bereich erneuerbarer Energien, die im Rahmen dieser Arbeit untersucht wird.



Ziel dieser Arbeit ist die Entwicklung einer
Web-Applikation, die als spezifisches Softwaretool für einen klar definierten
Anwendungsbereich dient. Neben der Umsetzung werden auch der Aufbau, die
Funktionsweise und das methodische Vorgehen beschrieben.



Die Firma, für die dieses Softwaretool entwickelt wird, ist
ein Serviceprovider für die Instandhaltung von Windenergieanlagen. Der
Teststand, an dem die Software eingesetzt werden soll, dient der Prüfung von
Umrichtern, die aus Windenergieanlagen stammen. Mittels eines firmeneigenen
Prüfverfahrens wird dabei festgestellt, ob diese Komponenten erneut in den
Anlagen verwendet werden können. Die im Rahmen der Tests erhobenen Messdaten
werden zusammen mit den zugehörigen Parametern und Informationen in Form von
XML-Dateien ausgegeben.



Die Applikation soll diese XML-Datensätze strukturiert in
einer Datenbank speichern und anschließend visuell aufbereiten. Die so
erzeugten Darstellungen dienen sowohl der internen Analyse als auch der
externen Kommunikation in Form von Kundenberichten. Dabei wird besonderer Wert
auf eine professionelle und leicht verständliche Visualisierung gelegt. Zudem
soll die Nutzeroberfläche intuitiv gestaltet und die Software modular
erweiterbar sein, sodass zukünftige Funktionen problemlos ergänzt werden
können. Ebenso wichtig ist eine einfache Integration in die bestehende
Systemlandschaft der Firma.



Um das gesetzte Ziel und die Anforderungen im Zeitrahmen
dieser Arbeit bestmöglich zu erfüllen, wird nach dem Erlangen und
Ausformulieren der theoretischen Grundlagen für die Erstellung der
Web-Applikation eine inkrementelle und iterative Vorgehensweise vorgesehen.
Diese soll für Flexibilität und hervorragendes Risikomanagement in Bezug auf
die Abgabe sorgen, da selbst bei Komplikationen eine funktionelle Version
vorliegt.



Die generellen Anforderungen und ihre Interpretation werden
in Kapitel \ref{sec:definition-der-anforderungen-und-ihre-interpretation}
behandelt. Das Kapitel \ref{sec:grundlagen} Das Kapitel „Grundlagen“
beschäftigt sich mit den theoretischen Hintergründen zur Entwicklung der
Applikation sowie mit den einzelnen zu implementierenden Funktionen. Gefolgt
von Kapitel \ref{sec:analyse-und-konzeption} „Analyse und Konzeption“, welches
sich mit dem bestehenden System und den einzufügenden Strukturen befassen. Das
dritte Kapitel \ref{sec:implementierung} „Implementierung“ beschreibt die
Entwicklung der Web-Applikation. Kapitel \ref{sec:integration-und-test}
„Integration und Test“ beschreibt die Einbindung in das bestehende System und
das Testen der Funktionalität. Abschließend Kapitel
\ref{sec:fazit-und-ausblick} „Fazit und Ausblick“ werden die Ergebnisse
zusammengefasst. und ein Ausblick auf mögliche Erweiterungen gegeben.













