%! Author = jbf
%! Date = 29.07.25

\usepackage[english,ngerman]{babel}		    %deutsche Trennmuster
\usepackage[utf8]{inputenc}					%direkte Umlauteingabe
\usepackage[T1]{fontenc}
\usepackage{mathptmx}	                    %Schriftpaket
\usepackage[scaled]{helvet}
\renewcommand{\familydefault}{\sfdefault}
\usepackage{array,ragged2e}					%wichtig für Abstandsformatierung
\usepackage{tocbasic} % KOMA für Verzeichnisse

\usepackage[automark,plainheadsepline=on]{scrlayer-scrpage}				%Anpassung von Kopf- und Fusszeilen
\usepackage{xspace}														%Korrektur Leerraum nach Befehlsdefinitionen
\usepackage{setspace}													%1,5 Zeilenabstand
\usepackage{graphicx}											        %zum Einbinden von Grafiken
\usepackage[absolute,overlay]{textpos}									%Boxen absolut Positionieren
\usepackage[final]{pdfpages}											%Einfügen von externen PDF Dokumenten
\usepackage{natbib}														%Neuimplementierung des \cite-Kommandos

\usepackage[
  top=3cm,      % Abstand oben
  left=3.5cm,   % Abstand links
  right=2cm,    % Abstand rechts
  bottom=2cm    % Abstand unten
]{geometry}

\usepackage{upgreek}            %Ermöglicht Aufrechte Varianten von Griechischen Buchstaben


 \usepackage[colorlinks,		% Einstellen und Laden des Hyperref-Pakets
	pdftex,
	bookmarks,
	bookmarksopen=false,
	bookmarksnumbered,
	citecolor=black,
	linkcolor=black,
	urlcolor=black,
	filecolor=black,
	linktocpage,
  pdfstartview=Fit,                  % startet mit Ganzseitenanzeige
	pdfsubject={Datenverarbeitung und Visualisierung von Umrichter-Testbench-Daten},
	pdftitle={Bachelorthesis im Fachbereich Elektrotechnik und Inforationstechnik an der FH Westküste},
	pdfauthor={Jon Feddersen}]{hyperref}

\usepackage{listings}


\setcounter{secnumdepth}{4}     %Nummerierung bis paragraph aktivieren
\setcounter{tocdepth}{4}        %Inhaltsverzeichnis zeigt bis \paragraph

\graphicspath{{Grafiken/}}      %Vereinfach das Einbinden von Grafiken da davon ausgegangen wird das alle in /Grafiken sind

\setlength{\parindent}{0pt} % Einrücken nach Absatzänderung

% --- Literaturverzeichnis-Paket ---
\usepackage[style=ieee,backend=biber]{biblatex}
\addbibresource{Literatur/Literatur.bib} % Bib-Datei mit Quellen